%% USPSC-Apendice.tex
% ---
% Inicia os apêndices
% ---

\begin{apendicesenv}
% Imprime uma página indicando o início dos apêndices
\partapendices
\chapter{Encoder 64b66b (74 bits com o CRC)}
\begin{lstlisting}
function [OUT_enc_74b,IN_esc_74b] = encoding64b66b(IN\_enc\_64b)

% -------"CODIFICADOR 64B/66B - VICTOR AFONSO DOS REIS - FEIS UNESP"-------

% -----Entradas e saídas---------------------
%IN_64b:[X0 X1 X2 ... X62 X63 X64]
%OUT_enc_74b: [Y1 Y2 Y3 ... Y72 Y73 Y74]
%OUT_2b: [Y0 Y1]

% Variáveis
persistent reg_scrambler;
OUT_enc_2b = zeros(1,2);
IN_esc_74b = zeros(1,74);
%% TRATAMENTO DA ENTRADA
%CRC
valCRC = 0;
msgIn = [zeros(1,8) IN_enc_64b]; % Adiciona 8 zeros no início do dado
regCRCenc = ones(1,8);

%% Inicialização das saídas
enc\_2b = dec2bin(1,2); % Seta para '01' pois só sera transmitido dado
inLength = length(enc_2b);
for i = 1:inLength
if strcmp(enc_2b(i),'0')
OUT_enc_2b(i) = 0;
else
OUT_enc_2b(i) = 1;
end
end

%CRC
OUT_line_72b = randi([0 0],1,72);

%% CRC - Polinômio G(x) = x^8 + x^2 + x + 1
lenCRC = length(msgIn);
for j = lenCRC:-1:1
valCRC = regCRCenc(8);
regCRCenc = circshift(regCRCenc,[0 1]);
regCRCenc(3) = bitxor(regCRCenc(3),valCRC);
regCRCenc(2) = bitxor(regCRCenc(2),valCRC);
regCRCenc(1) = bitxor(valCRC,msgIn(j));
end
finalXOR = ones(1,8);
regxor = bitxor(regCRCenc,finalXOR);

OUT_line_72b = [IN_enc_64b regxor];
IN_esc_74b = [OUT_enc_2b OUT_line_72b];
%% Scrambler - Polinômio G(x) = x^58 + x^39 + 1
if isempty(reg_scrambler)
reg_scrambler = ones(1,58);
end
val = zeros(1,72);
v = randi([0 0],1,1);
for j = 1:72
v = bitxor(reg_scrambler(58),reg_scrambler(39));
v = bitxor(v, OUT_line_72b(j));
reg_scrambler = circshift(reg_scrambler, [0 1]);
reg_scrambler(1) = v;
val(j) = v;
end

OUT_enc_74b = [OUT_enc_2b val];
end
\end{lstlisting}
    


% ----------------------------------------------------------
\chapter{Siglas dos Programas de Pós-Graduação do IAU}
\index{quadros}O \autoref{quadro-iau} relaciona as siglas estabelecidas para os programas de pós-graduação do IAU.
\begin{quadro}[htb]
\ABNTEXfontereduzida
\caption[Siglas dos Programas de Pós-Graduação do IAU]{Siglas dos Programas de Pós-Graduação do IAU}
\label{quadro-iau}
\begin{tabular}{|p{3.5cm}|p{3.5cm}|p{3.5cm}|p{1.5cm}|p{2.25cm}|}
  \hline
   \textbf{PROGRAMA} & \textbf{ÁREA DE CONCENTRAÇÃO} & \textbf{OPÇÃO} & \textbf{TÍTULO} & \textbf{SIGLA}  \\
    \hline
Programa de Pós-Graduação em Arquitetura e Urbanismo & Arquitetura, Urbanismo e Tecnologia &  & Doutor & DAUT\\
Programa de Pós-Graduação em Arquitetura e Urbanismo & Arquitetura, Urbanismo e Tecnologia &  & Mestre & MAUT\\
Programa de Pós-Graduação em Arquitetura e Urbanismo & Teoria e História da Arquitetura e do Urbanismo &  & Doutor & DAUH\\
Programa de Pós-Graduação em Arquitetura e Urbanismo & Teoria e História da Arquitetura e do Urbanismo &  & Mestre & MAUH\\
    \hline

\end{tabular}
\begin{flushleft}
		Fonte: Elaborado pelos autores.\
\end{flushleft}
\end{quadro}

% ----------------------------------------------------------
\chapter{Siglas dos Programas de Pós-Graduação do ICMC}
\index{quadros}O \autoref{quadro-icmc} relaciona as siglas estabelecidas para os programas de pós-graduação do ICMC.
\begin{quadro}[htb]
\ABNTEXfontereduzida
\caption[Siglas dos Programas de Pós-Graduação do ICMC]{Siglas dos Programas de Pós-Graduação do ICMC}
\label{quadro-icmc}
\begin{tabular}{|p{3.5cm}|p{3.5cm}|p{3.5cm}|p{1.5cm}|p{2.25cm}|}
  \multicolumn{5}{r}{{(continua)}} \\ 
  \hline
   \textbf{PROGRAMA} & \textbf{ÁREA DE CONCENTRAÇÃO} & \textbf{OPÇÃO} & \textbf{TÍTULO} & \textbf{SIGLA}  \\
    \hline
		Ciências de Computação e Matemática Computacional	& Ciências de Computação e Matemática Computacional	&   &	Doutor	 & DCCp\\
    Ciências de Computação e Matemática Computacional	& Ciências de Computação e Matemática Computacional	&   &	Mestre	& MCCp\\
		Computer Science and Computational Mathematics & Computer Science and Computational Mathematics	&   &	Doctorate & DCCe\\
		Doctorate Program in Mathematics & Mathematics &   &	Doctorate & DMAe\\
		Interinstitucional de Pós-Graduação em Estatística & Estatística &  & Doutor	 & DESp\\
		Interinstitucional de Pós-Graduação em Estatística & Estatística &  & Mestre & MESp\\
		Join Graduate Program in Statistics & Computer Science and Computational Mathematics &  & Master & MCCe\\
		Join Graduate Program in Statistics & Statistics &  & Doctorate & 	DESe\\
    Join Graduate Program in Statistics & Statistics &  & Master & MESe\\
		Master Program in Mathematics &	Mathematics &  & Master &	MMAe\\
		Mathematics Professional Master\'{}s Program &	Mathematics &	 & Master &	MPMe\\
		Programa de Mestrado Profissional em Matemática & Matemática &  & Mestre & MPMp\\
		\end{tabular}
\end{quadro}

% o comando \clearpage é necessário para deixar o final da tabela o topo da página, sem ele o final da tabela é centralizado verticalmente na página 
\clearpage
\begin{quadro}[htb]
\ABNTEXfontereduzida
\begin{tabular}{|p{3.5cm}|p{3.5cm}|p{3.5cm}|p{1.5cm}|p{2.25cm}|}
	\multicolumn{5}{c}%
	{{\quadroname\ \thequadro{} -- Siglas dos Programas de Pós-Graduação do ICMC}} \\
	\multicolumn{5}{r}{{(conclusão)}} \\
	\hline
   \textbf{PROGRAMA} & \textbf{ÁREA DE CONCENTRAÇÃO} & \textbf{OPÇÃO} & \textbf{TÍTULO} & \textbf{SIGLA}  \\	
	 \hline
  	Programa de Pós-Graduação em Matemática & Matemática &  & Doutor & DMAp\\
		Programa de Pós-Graduação em Matemática & Matemática &  & Mestre & MMAp\\
    \hline

\end{tabular}
\begin{flushleft}
		Fonte: Elaborado pelos autores.\
\end{flushleft}
\end{quadro}

% ----------------------------------------------------------
\chapter{Siglas dos Programas de Pós-Graduação do IFSC}
\index{quadros}O \autoref{quadro-ifsc} relaciona as siglas estabelecidas para os programas de pós-graduação do IFSC.
\begin{quadro}[htb]
\ABNTEXfontereduzida
\caption[Siglas dos Programas de Pós-Graduação do IFSC]{Siglas dos Programas de Pós-Graduação do IFSC}
\label{quadro-ifsc}
\begin{tabular}{|p{3.5cm}|p{3.5cm}|p{3.5cm}|p{1.5cm}|p{2.25cm}|}
  \hline
   \textbf{PROGRAMA} & \textbf{ÁREA DE CONCENTRAÇÃO} & \textbf{OPÇÃO} & \textbf{TÍTULO} & \textbf{SIGLA}  \\
    \hline
Graduate Program in Physics & Applied Physics & Biomolecular Physics & Doutor & DFAFBe\\
Programa de Pós-Graduação do Instituto de Física de São Carlos & Física Aplicada &  & Doutor & DFA\\
Programa de Pós-Graduação do Instituto de Física de São Carlos & Física Aplicada & Física Computacional & Doutor & DFAFC\\
Programa de Pós-Graduação do Instituto de Física de São Carlos & Física Aplicada & Física Biomolecular & Doutor & DFAFBp\\
Programa de Pós-Graduação do Instituto de Física de São Carlos & Física Aplicada &  & Mestre & MFA\\
Programa de Pós-Graduação do Instituto de Física de São Carlos & Física Aplicada & Física Computacional & Mestre & MFAFC\\
Programa de Pós-Graduação do Instituto de Física de São Carlos & Física Aplicada & Física Biomolecular & Mestre & MFAFB\\
Programa de Pós-Graduação do Instituto de Física de São Carlos & Física Básica &  & Doutor & DFB\\
Programa de Pós-Graduação do Instituto de Física de São Carlos & Física Básica &  & Mestre & MFB\\
		\hline

\end{tabular}
\begin{flushleft}
		Fonte: Elaborado pelos autores.\
\end{flushleft}
\end{quadro}

% ----------------------------------------------------------
\chapter{Siglas dos Programas de Pós-Graduação do IQSC}
\index{quadros}O \autoref{quadro-iqsc} relaciona as siglas estabelecidas para os programas de pós-graduação do IQSC.
\begin{quadro}[htb]
\ABNTEXfontereduzida
\caption[Siglas dos Programas de Pós-Graduação do IQSC]{Siglas dos Programas de Pós-Graduação do IQSC}
\label{quadro-iqsc}
\begin{tabular}{|p{3.5cm}|p{3.5cm}|p{3.5cm}|p{1.5cm}|p{2.25cm}|}
  \hline
   \textbf{PROGRAMA} & \textbf{ÁREA DE CONCENTRAÇÃO} & \textbf{OPÇÃO} & \textbf{TÍTULO} & \textbf{SIGLA}  \\
    \hline
Programa de Pós-Graduação do Instituto de Química de São Carlos & Físico-química &  & Doutor & DFQ\\
Programa de Pós-Graduação do Instituto de Química de São Carlos & Físico-química &  & Mestre & MFQ\\
Programa de Pós-Graduação do Instituto de Química de São Carlos & Química Analítica e Inirgânica &  & Doutor & DQAI\\
Programa de Pós-Graduação do Instituto de Química de São Carlos & Química Analítica e Inirgânica &  & Mestre & MQAI\\
Programa de Pós-Graduação do Instituto de Química de São Carlos & Química Orgânica e Biológica &  & Doutor & DQOB\\
Programa de Pós-Graduação do Instituto de Química de São Carlos & Química Orgânica e Biológica &  & Mestre & MQOB\\
\hline

\end{tabular}
\begin{flushleft}
		Fonte: Elaborado pelos autores.\
\end{flushleft}
\end{quadro}


% ----------------------------------------------------------
\chapter{Siglas dos Cursos de Graduação da EESC}
\index{quadros}O \autoref{quadro-geesc} relaciona as siglas estabelecidas para os cursos de graduação da EESC.
\begin{quadro}[htb]
	\ABNTEXfontereduzida
	\caption[Siglas dos Cursos de Graduação da EESC]{Siglas dos Cursos de Graduação da EESC}
	\label{quadro-geesc}
	\begin{tabular}{|p{6.5cm}|p{6.5cm}|p{1.75cm}|}
		\hline
		\textbf{CURSO} & \textbf{TÍTULO} &  \textbf{SIGLA}  \\
		\hline
		Engenharia Ambiental & Engenheiro Ambiental & EAMB\\
		Engenharia Aeronáutica & Engenheiro Aeronáutico & EAER\\
		Engenharia Civil & Engenheiro Civil & ECIV\\
		Engenharia de Computação & Engenheiro de Computação & ECOM\\
	    Engenharia Elétrica com Ênfase em Eletrônica & Engenheiro Eletricista & EELT\\
	    Engenharia Elétrica com Ênfase em Sistemas de Energia e Automação & Engenheiro Eletricista & EELS\\
		Engenharia de Materiais e Manufatura & Engenheiro de Materiais e de Manufatura & EMAT\\
		Engenharia Mecânica & Engenheiro Mecatrônico & EMET\\
		Engenharia de Produção & Engenheiro de Produção & EPRO\\
		\hline
		
	\end{tabular}
	\begin{flushleft}
		Fonte: Elaborado pelos autores.\
	\end{flushleft}
\end{quadro}

% ----------------------------------------------------------
\chapter{Exemplo de tabela centralizada verticalmente e horizontalmente}
\index{tabelas}A \autoref{tab-centralizada} exemplifica como proceder para obter uma tabela centralizada verticalmente e horizontalmente.
% utilize \usepackage{array} no PREAMBULO (ver em USPSC-modelo.tex) obter uma tabela centralizada verticalmente e horizontalmente
\begin{table}[htb]
\ABNTEXfontereduzida
\caption[Exemplo de tabela centralizada verticalmente e horizontalmente]{Exemplo de tabela centralizada verticalmente e horizontalmente}
\label{tab-centralizada}

\begin{tabular}{ >{\centering\arraybackslash}m{6cm}  >{\centering\arraybackslash}m{6cm} }
\hline
 \centering \textbf{Coluna A} & \textbf{Coluna B}\\
\hline
  Coluna A, Linha 1 & Este é um texto bem maior para exemplificar como é centralizado verticalmente e horizontalmente na tabela. Segundo parágrafo para verificar como fica na tabela\\
  Quando o texto da coluna A, linha 2 é bem maior do que o das demais colunas  & Coluna B, linha 2\\
\hline
\end{tabular}
\begin{flushleft}
		Fonte: Elaborada pelos autores.\
\end{flushleft}
\end{table}

% ----------------------------------------------------------
\chapter{Exemplo de tabela com grade}
\index{tabelas}A \autoref{tab-grade} exemplifica a inclusão de traços estruturadores de conteúdo para melhor compreensão do conteúdo da tabela, em conformidade com as normas de apresentação tabular do IBGE.
% utilize \usepackage{array} no PREAMBULO (ver em USPSC-modelo.tex) obter uma tabela centralizada verticalmente e horizontalmente
\begin{table}[htb]
\ABNTEXfontereduzida
\caption[Exemplo de tabelas com grade]{Exemplo de tabelas com grade}
\label{tab-grade}
\begin{tabular}{ >{\centering\arraybackslash}m{8cm} | >{\centering\arraybackslash}m{6cm} }
\hline
 \centering \textbf{Coluna A} & \textbf{Coluna B}\\
\hline
  A1 & B1\\
\hline
  A2 & B2\\
\hline
  A3 & B3\\
\hline
  A4 & B4\\
\hline
\end{tabular}
\begin{flushleft}
		Fonte: Elaborada pelos autores.\
\end{flushleft}
\end{table}


\end{apendicesenv}
% ---