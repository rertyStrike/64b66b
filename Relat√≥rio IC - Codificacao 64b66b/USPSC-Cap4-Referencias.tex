% ---
%% USPSC-Cap4-Referencias.tex
% --
% Este capítulo traz os exemplos de referências das "Diretrizes para apresentação de dissertações e teses da USP: documento eletrônico e impresso - Parte I (ABNT)" disponílvel em: http://biblioteca.puspsc.usp.br/pdfFiles_Caderno_Estudos_9_PT_1.pdf


% --- 
\chapter{Modelos de referências}
\label{Referências}
% --- 
Elemento obrigatório, que consiste na relação das obras consultadas e citadas no texto, de maneira que permita a identificação individual de cada uma delas. As referências devem ser organizadas em ordem alfabética, caso as citações no texto obedeçam ao sistema autor-data, ou conforme aparecem no texto, quando utilizado o sistema numérico de chamada. \cite{sibi2009}.

O capítulo 4 sobre referências foi elaborado com base nas \textbf{Diretrizes para apresentação de dissertações e teses da USP}: documento eletrônico e impresso - Parte I (ABNT) e todos os exemplos aqui apresentados constam dessas Diretrizes.  

Para organização, gerenciamento e editoração das referências em BibTeX foi utilizado o software JabRef versão 2.10.

A ABNT NBR 6023 especifica os elementos a serem incluídos, fixa sua ordem, orienta a preparação e compilação das referências de materiais utilizados para a produção de documentos e para a inclusão em bibliografias, resumos etc. \cite{nbr6023}.

Normalmente não há problemas em usar caracteres acentuados em arquivos bibliográficos {(*.bib)}. Porém, como as regras da ABNT 6023 exigem a conversão do autor ou organização para letras maiúsculas, é preciso observar o modo como se escrevem os nomes dos autores. No ~\autoref{quadro-acentos} você encontra alguns
exemplos das conversões mais importantes. Preste atenção especial para `ç' e `í'
que devem estar envoltos em chaves. A regra geral é sempre usar a acentuação neste modo quando houver conversão para letras maiúsculas. \cite{abnetxcite} \\

\begin{quadro}[H]
	\caption{\label{quadro-acentos}Conversão de acentuação}
		\begin{tabular}{|p{7.5cm}|p{7.5cm}|}
			\hline
			\textbf{Acentos} & \textbf{BibTeX}\\
			\hline
			à á ã & \verb+\`a+ \verb+\'a+ \verb+\~a+\\
			\hline
			í & \verb+{\'\i}+\\
			\hline
			ç & \verb+{\c c}+\\
			\hline
		\end{tabular}
		\begin{flushleft}
			Fonte: \citeonline{abnetxcite}
		\end{flushleft}	
\end{quadro}


\section{Monografias}

Livros, folhetos, guias, catálogos, fôlderes, dicionários e trabalhos acadêmicos.

Elementos essenciais: autoria, título, edição, local de publicação, editora e ano de publicação.
Elementos complementares: responsabilidade (tradutor, revisor, ilustrador, entre outros), paginação, série, notas e ISBN.

O prenome pode estar abreviado ou por extenso, porém deve estar padronizado em toda a listagem. \\

\subsection{Monografia no todo}

\begin{tabular}{|l|c|} \hline
SOBRENOME, Prenome(s) do(s) autor(es). \textbf{Título da obra}: subtítulo.\\Edição. 
Local:	Editora, data de publicação. Paginação. Série. Notas. ISBN.\\\hline
\end{tabular}\\

\subsubsection{Um autor}

\begin{tabular}{|l|c|} \hline
 ESPÍRITO SANTO, A. \textbf{Essências de metodologia científica}: aplicada \\
 à educação. Londrina: Universidade Estadual, 1987. \\\hline
\end{tabular}\\

\textbf{Campos em LATEX:}

\begin{verbatim}
\@Book{EspiritoSanto1987,
Title                    = {Essências de metodologia científica},
Address                  = {Londrina},
Author                   = {Esp{\'\i}rito, Santo, A.},
Publisher                = {Universidade Estadual},
Subtitle                 = {aplicada à educação},
Year                     = {1987},
Owner                    = {apcalabrez},
Timestamp                = {2015.09.21}
}
\end{verbatim}

\begin{tabular}{|l|c|} \hline
PICCINI, A. \textbf{Cortiços na cidade}: conceito e preconceito na reestruturação\\ do centro urbano de São Paulo. São Paulo: Annablume, 1999. 166 p. \\\hline
\end{tabular}\\

\textbf{Campos em LATEX:}

\begin{verbatim}
@Book{Piccini1999,
Title                    = {Cortiços na cidade},
Address                  = {São Paulo},
Author                   = {Piccini, A.},
Pages                    = {166},
Publisher                = {Annablume},
Subtitle                 = {conceito e preconceito na reestruturação
do centro urbano de São Paulo},
Year                     = {1999},
Owner                    = {apcalabrez},
Timestamp                = {2015.09.21}
}
\end{verbatim}

\subsubsection{Dois autores}

\begin{tabular}{|l|c|} \hline
NOVAK, E.R; WOODRUFF, J. D. \textbf{Novak's ginecologic and obstetric}\\ \textbf{pathology.} Philadelphia: Saunders, 1967. \\\hline
\end{tabular}\\

\textbf{Campos em LATEX:}
\begin{verbatim}
@Book{Novak1967,
Title                    = {Novak's ginecologic and obstetric 
pathology},
Address                  = {Philadelphia},
Author                   = {Novak, E. R. and Woodruff, J. D.},
Publisher                = {Saunders},
Year                     = {1967},
Owner                    = {apcalabrez},
Timestamp                = {2015.09.21}
}

\end{verbatim}

\begin{tabular}{|l|c|} \hline
	GOMES, C. B.; KEIL, K. \textbf{Brazilian stone meteorites.} 
	Albuquerque: \\ University of New Mexico, 1980. \\\hline
\end{tabular}\\

\textbf{Campos em LATEX:}

\begin{verbatim}
@Book{Gomes1980,
Title                    = {Brazilian stone meteorites},
Address                  = {Albuquerque},
Author                   = {Gomes, C. B. and Keil, K.},
Publisher                = {University of New Mexico},
Year                     = {1980},
Owner                    = {apcalabrez},
Timestamp                = {2015.09.21}
}
\end{verbatim}

\subsubsection{Três autores}

\begin{tabular}{|l|c|} \hline
GIANNINI, S. D.; FORTI, N.; DIAMENT, J. \textbf{Cardiologia preventiva}:\\ prevenção primária e secundária. São Paulo: Atheneu, 2000. \\\hline
\end{tabular}\\
\\
\textbf{Campos em LATEX:}

\begin{verbatim}
@Book{Giannini2000,
Title                    = {Cardiologia preventiva},
Address                  = {São Paulo},
Author                   = {Giannini, S. D. and Forti, N. and Diament, 
J.},
Publisher                = {Atheneu},
Subtitle                 = {prevenção primária e secundária},
Year                     = {2000},
Owner                    = {apcalabrez},
Timestamp                = {2015.09.21}
}

\end{verbatim}

\begin{tabular}{|l|c|} \hline
GLASSCOCK III, M. E.; JACKSON, C. G.; JOSEY, A. F. \textbf{Abr handbook}: \\ auditory brainstem response. 2nd ed. New York: Tieme Medical, 1987. \\\hline
\end{tabular}\\

\textbf{Campos em LATEX:}

\begin{verbatim}
@Book{GlasscockIII1987,
Title                    = {Abr handbook},
Address                  = {New York},
Author                   = {Glasscock, III, M. E. and Jackson, 
C. G. and Josey, A. F.},
Publisher                = {Tieme Medical},
Subtitle                 = {auditory brainstem response},
Year                     = {1987},
Edition                  = {2nd},
Owner                    = {apcalabrez},
Timestamp                = {2015.09.21}
}
\end{verbatim}

\subsubsection{Quatro autores}

\begin{tabular}{|l|c|} \hline
PASQUARELLI, M. L. R. et al. \textbf{Avaliação do uso de periódicos}. 
São \\ Paulo: SIBi-USP, 1987. 14 p.\\\hline
\end{tabular}\\

\textbf{Campos em LATEX:}

\begin{verbatim}
@Book{Pasquarelli1987,
Title                    = {Avaliação do uso de periódicos},
Address                  = {São Paulo},
Author                   = {Pasquarelli, M. L. R. and Krzyzanowski,
R. F.; Imperatriz, I. M. M.; Noronha, D. P.; Andrade, E.; Zapparoli,
M. C. M.; Bonesio, M. C. M.; Lobo, M. P.; Almeida, M. S.; Arruda, 
R. M. A.; Plaza, R. T. T.},
Pages                    = {14},
Publisher                = {SIBi-USP},
Year                     = {1987},
Owner                    = {apcalabrez},
Timestamp                = {2015.09.21}
}
\end{verbatim}

\textbf{Nota:} é facultada a indicação de todos os autores para casos específicos, tais como: projetos de pesquisa científica e indicação de
produção científica em relatórios para órgãos de financiamento. 

Para desativar a substituição dos autores por ‘et al.’, nas referências você deve incluir o pacote com a seguinte opção: \verb+\usepackage[alf,abnt-etal-cite=0]{abntex2cite}+

No ~\autoref{quadro-opcoes-etal} estão descritos os comandos dos pacotes de alteração da composição dos estilos bibliográficos para alterar o estilo ‘et al.’

\begin{quadro}[H]
	\caption{\label{quadro-opcoes-etal}Opções de alteração da composição dos estilos bibliográficos para utilização da sigla ‘et al.’}
		\begin{tabular}{|p{4.0cm}|p{2.0cm}|p{8.5cm}|}
			\hline
			\textbf{Campo} & \textbf{Opções} & \textbf{Descrição} \\ 
			\hline
			\emph{abnt-etal-cite} &  & controla como e quando os co-autores são
			substituídos por \emph{et al.}.  Note que a substituição
			por \emph{et al.} continua ocorrendo \emph{sempre} se os co-autores tiverem sido indicados
			como \texttt{others}.\\
			\hline
			\texttt{abnt-etal-cite=0}&\texttt{0}& não abrevia a lista de autores.\\
			\hline
			\texttt{abnt-etal-cite=2}& \texttt{2} & abrevia com mais de 2 autores.\\
			\hline
			\texttt{abnt-etal-cite=3}& \texttt{3} & abrevia com mais de 2 autores.\\
			\hline
			$\vdots$ & $\vdots$ & \\
			\hline
			\texttt{abnt-etal-cite=5}& \texttt{5} & abrevia com mais de 5 autores.\\
			\hline
		\end{tabular}
	\begin{flushleft}
		Fonte: \citeonline{abnetxcite}
	\end{flushleft}	
\end{quadro}

Para ver as demais opções e o modo de uso dos pacotes de especificidades para formatação de referências veja o documento \textbf{O pacote abntex2cite}. \cite{abnetxcite}.

Sendo assim, para que todos os nomes dos autores constem da referência basta acrescentar o pacote: 

\verb+\usepackage[alf,abnt-etal-cite=0]{abntex2cite}+

E a referência será escrita da seguinte forma: \\

\begin{tabular}{|l|c|} \hline
PASQUARELLI, M. L. R.; KRZYZANOWSKI, R. F.; IMPERATRIZ, I. M.\\
M.; NORONHA, D. P.; ANDRADE, E.; ZAPPAROLI, M. C. M.; BONESIO, \\
M. C. M.; LOBO, M. P.; ALMEIDA, M. S.; ARRUDA, R. M. A.; PLAZA, R. \\ \textbf{Avaliação do uso de periódicos}. São Paulo: SIBi-USP, 1987. 14 p.\\\hline
\end{tabular}\\

\textbf{Campos em LATEX:} permanecerão transcritos da mesma forma.\\

\begin{verbatim}
@Book{Pasquarelli1987,
Title                    = {Avaliação do uso de periódicos},
Address                  = {São Paulo},
Author                   = {Pasquarelli, M. L. R. and Krzyzanowski,
R. F.; Imperatriz, I. M. M.; Noronha, D. P.; Andrade, E.; Zapparoli,
M. C. M.; Bonesio, M. C. M.; Lobo, M. P.; Almeida, M. S.; Arruda, 
R. M. A.; Plaza, R. T. T.},
Pages                    = {14},
Publisher                = {SIBi-USP},
Year                     = {1987},
Owner                    = {apcalabrez},
Timestamp                = {2015.09.21}
}
\end{verbatim}

\subsubsection{Autoria Desconhecida}

\begin{tabular}{|l|c|} \hline
A BETTER investiment climate for everyone. Washington: Oxford University \\ Press, 2004.\\\hline
\end{tabular}\\

\textbf{Campos em LATEX:}

\begin{verbatim}
@Book{abetter2004,
Title                    = {A BETTER investiment climate for everyone},
Address                  = {Washington},
Org-short                = {A Better},
Publisher                = {Oxford University Press},
Year                     = {2004},
Owner                    = {apcalabrez},
Timestamp                = {2015.09.21}
}
\end{verbatim}

\begin{tabular}{|l|c|} \hline
EDUCAÇÃO para todos: o imperativo da qualidade. Brasília, DF: Unesco,\\ 2005.\\\hline
\end{tabular}\\

\textbf{Campos em LATEX:}

\begin{verbatim}
@Book{educacao2005,
Title                    = {Educa{\c c}\~ao para todos},
Address                  = {Brasília, DF},
Org-short                = {Educa{\c c}\~ao},
Publisher                = {Unesco},
Subtitle                 = {o imperativo da qualidade},
Year                     = {2005},
Owner                    = {apcalabrez},
Timestamp                = {2015.09.21}
}
\end{verbatim}

\subsubsection{Tradutor, prefaciador, ilustrador, compilador, revisor}

\begin{tabular}{|l|c|} \hline
FONSECA, R. J. (Ed.). \textbf{Oral and maxillofacial surgery}. Illustrated by\\
William M. Winn. Philadelphia: Saunders, 2000. \\\hline
\end{tabular}\\

\textbf{Campos em LATEX:}

\begin{verbatim}
@Book{Fonseca2000,
Title                    = {Oral and maxillofacial surgery},
Address                  = {Philadelphia},
Editor                   = {Fonseca, R. J.},
Furtherresp              = {llustrated by William M. Winn},
Publisher                = {Saunders},
Year                     = {2000},
Owner                    = {apcalabrez},
Timestamp                = {2015.09.17}
}
\end{verbatim}

\begin{tabular}{|l|c|} \hline
GOMES, A. C.; VECHI, C. A. \textbf{Estática romântica}: textos doutrinários
\\ comentados. Tradução de Maria Antonia Simões Nunes, Duílio Colombini.
\\São Paulo: Atlas, 1992. 186 p.  \\\hline
\end{tabular}\\

\textbf{Campos em LATEX:}

\begin{verbatim}
@Book{Gomes,
Title                    = {Estática romântica},
Address                  = {São Paulo},
Author                   = {Gomes, A. C. and Vechi, C. A.},
Furtherresp              = {Tradução de Maria Antonia Simões Nunes, 
Duílio Colombini},
Pages                    = {186},
Publisher                = {Atlas},
Subtitle                 = {textos doutrinários},
Year                     = {1992},
Owner                    = {apcalabrez},
Timestamp                = {2015.09.17}
}
\end{verbatim}

\begin{tabular}{|l|c|} \hline
SAADI, S. \textbf{O jardim das rosas}. Tradução de Aurélio Buarque de Holanda.\\ Rio de Janeiro: J. Olympio, 1944. 124 p., il. (Coleção Rubayat).Versão francesa\\ de Franz Toussaint do original árabe.  \\\hline
\end{tabular}\\

\textbf{Campos em LATEX:}

\begin{verbatim}
@Book{Saadi1944,
Title                    = {O jardim das rosas},
Address                  = {Rio de Janeiro},
Author                   = {Saadi, S.},
Furtherresp              = {Tradução de Aurélio Buarque de Holanda},
Note                     = {Versão francesa de Franz Toussaint do 
original árabe},
Pages                    = {124},
Publisher                = {J. Olympio},
Series                   = {Coleção Rubayat},
Year                     = {1944},
Owner                    = {apcalabrez},
Timestamp                = {2015.09.17}
}
\end{verbatim}

\subsubsection{Série}

\begin{tabular}{|l|c|} \hline
PHILLIPI JÚNIOR, A. et al. \textbf{Interdisciplinaridade em ciências ambien-}\\ 
\textbf{tais}. São Paulo: Signus, 2000. 318 p. (Série textos básicos para a formação \\ambiental, 5). \\\hline
\end{tabular}\\

\textbf{Campos em LATEX:}

\begin{verbatim}
@Book{PhillipiJunior2000,
Title                 = {Interdisciplinaridade em ciências ambientais},
Address               = {São Paulo},
Author                = {Phillipi, Junior, A. and Medeiros, C. B. and 
Silva, A. M. and Piccini, A.},
Pages                 = {318},
Publisher             = {Signus},
Series                = {Série textos básicos para a formação ambiental, 
5},
Year                  = {2000},
Owner                 = {apcalabrez},
Timestamp             = {2015.09.21}
}
\end{verbatim}

\subsubsection{Editor, organizador, coordenador etc.}

\begin{tabular}{|l|c|} \hline
DEL VECCHIO, M. (Comp.). \textbf{A Vista de antejo longa mira}: los \\antejos
del  Luxottica, as lunetas do Museo Luxottica. Tradução de G. Lizabe \\M. Maglione,  Monique Di Prima. Milão: Arti Grafiche Salea Luxottica, 1995.  \\\hline
\end{tabular}\\

\textbf{Campos em LATEX:}

\begin{verbatim}
@Book{delvecchio1995,
Title                    = {A Vista de antejo longa mira},
Address                  = {Milão},
Editor                   = {Del, Vecchio, M},
Editortype               = {Comp.},
Furtherresp              = {Tradução de G. Lizabe M. Maglione, Monique 
Di Prima},
Publisher                = {Arti Grafiche Salea Luxottica},
Subtitle                 = {los antejos del Luxottica, as lunetas do 
Museo Luxottica.},
Year                     = {1995},
Owner                    = {apcalabrez},
Timestamp                = {2015.09.21}
}
\end{verbatim}

\begin{tabular}{|l|c|} \hline
PLOTKIN, S. A.; ORENSTEIN, W. A. (Ed.). \textbf{Vaccines}. 3rd ed. Philadelphia: \\ W.B. Saunders, 1999. 1230 p.  \\\hline
\end{tabular}\\

\textbf{Campos em LATEX:}

\begin{verbatim}
@Book{Plotkin1999,
Title                    = {Vaccines.},
Address                  = {Philadelphia},
Editor                   = {Plotkin, S. A. and Orenstein W. A.},
Editortype               = {Ed.},
Pages                    = {1230},
Publisher                = {W.B. Saunders},
Year                     = {1999},
Edition                  = {3rd ed},
Owner                    = {apcalabrez},
Timestamp                = {2016.03.31}
}
\end{verbatim}

\begin{tabular}{|l|c|} \hline
TORTAMANO, N. (Coord.). \textbf{G.T.O.}: guia terapêutico odontológico. 8. ed. \\São  Paulo: EBO, 1989. 248 p.  \\\hline
\end{tabular}\\

\textbf{Campos em LATEX:}

\begin{verbatim}
@Book{Tortamano1989,
Title                    = {G.T.O.},
Address                  = {São Paulo},
Editor                   = {Tortamano, N.},
Editortype               = {Coord.},
Pages                    = {248},
Publisher                = {EBO},
Subtitle                 = {guia terapêutico odontológico},
Year                     = {1989},
Edition                  = {8. ed.},
Owner                    = {apcalabrez},
Timestamp                = {2015.09.22}
}
\end{verbatim}

\subsubsection{Autor e editor}

\begin{tabular}{|l|c|} \hline
HENNEKENS, C. H.; BURING, J. E. \textbf{Epidemiology in medicine}. Phila-\\delphia:  Lippincott Williams e Wilkins, 1987. 383 p. Edited by Sherry L. \\Mayrent. \\\hline
\end{tabular}\\

\textbf{Campos em LATEX:}

\begin{verbatim}
@Book{Hennekens1987b,
Title                    = {Epidemiology in medicine},
Address                  = {Philadelphia},
Author                   = {Hennekens, C. H. and Buring, J. E.},
Note                     = {Edited by Sherry L. Mayrent},
Pages                    = {383},
Publisher                = {Lippincott Williams \& Wilkins},
Year                     = {1987},
Owner                    = {apcalabrez},
Timestamp                = {2015.09.22}
}
\end{verbatim}
\subsubsection{Pseudônimo}

Deve ser adotado na referência, desde que seja a forma adotada pelo autor. \\

\begin{tabular}{|l|c|} \hline
ATHAYDE, Tristão de. \textbf{Debates pedagógicos}. Rio de Janeiro: Schmidt, \\ 1931. 180 p.   \\\hline
\end{tabular}\\

\textbf{Campos em LATEX:}

\begin{verbatim}
@Book{Athayde1931,
Title                    = {Debates pedagógicos},
Address                  = {Rio de Janeiro},
Author                   = {Athayde, Tristão de},
Pages                    = {180},
Publisher                = {Schmidt},
Year                     = {1931},
Owner                    = {apcalabrez},
Timestamp                = {2016.03.31}
}
\end{verbatim}

\subsubsection{Autor entidade (entidades coletivas, governamentais, públicas, particulares etc.) }

As obras de responsabilidade de autor entidade (órgãos governamentais, empresas, associações, comissões, congressos,
seminários etc.) têm entrada pelo próprio nome da entidade, por extenso.

Seu nome é precedido pelo nome do órgão superior, ou pelo nome da jurisdição geográfica à qual pertence.  

No capítulo \ref{Citações} foram exemplificados algumas citações com  as referências para entidades coletivas. Conforme exposto anteriormente os arquivos.bib de referências para entidade coletiva deve conter o comando Org-short que equivale a forma como à referência será citada no texto. \\

\begin{tabular}{|l|c|} \hline
AGÊNCIA NACIONAL DE VIGILÂNCIA SANITÁRIA. \textbf{Política vigente} \\ 
\textbf{para a regulamentação de medicamentos no Brasil}. Brasília, DF, 2003.  \\\hline
\end{tabular}\\

\textbf{Campos em LATEX:} 

\begin{verbatim}
@Book{Agencia2003,
Title                    = {Política vigente para a regulamentação de 
medicamentos no Brasil},
Address                  = {Brasília, DF},
Org-short                = {Ag\^encia Nacional de Vigil\^ancia Sani
t\'aria},
Organization             = {Ag\^encia Nacional de Vigil\^ancia Sani
t\'aria},
Year                     = {2003},
Owner                    = {apcalabrez},
Timestamp                = {2015.09.22}
}
\end{verbatim}

Para este exemplo a citação será por extenso: \cite{Agencia2003}. 

Par a unidade que desejar inserir na citação a sigla da entidade coletiva deverá preencher o campo Org-short com a sigla da entidade. \\

\begin{tabular}{|l|c|} \hline
	UNIVERSIDADE DE SÃO PAULO. Sistema Integrado de Bibliotecas. \\Departamento Técnico.  \textbf{Bibliotheca universitatis}: livros impressos \\dos séculos XV e XVI do acervo bibliográfico da Universidade de São \\Paulo. São Paulo: EDUSP, 2000. 705 p.   \\\hline
\end{tabular}\\

\textbf{Campos em LATEX:}

\begin{verbatim}
@Book{usp2000,
Title                    = {Bibliotheca universitatis},
Address                  = {São Paulo},
Org-short                = {USP},
Organization             = {Universidade de S\~ao Paulo. {Sistema 
Integrado de Bibliotecas. Departamento Técnico}},
Pages                    = {705},
Publisher                = {EDUSP},
Subtitle                 = {livros impressos dos séculos XV e XVI 
do acervo bibliográfico da Universidade de São Paulo},
Year                     = {2000},
Owner                    = {apcalabrez},
Timestamp                = {2015.09.23}
}
\end{verbatim}

Para este exemplo a citação será pela sigla do órgão superior: \cite{usp200}. \\

De acordo com norma de citações da ABNT NBR 10520 a entrada da citação deverá ser pelo "nome de cada entidade responsável até o primeiro sinal de pontuação". \cite{nbr10520}. 

Para entidades coletivas que possuírem órgão superior ou jurisdição geográfica deverá ser inserido no campo Org-short o nome do órgão superior  ou jurisdição geográfica e no campo Organization o nome completo da entidade coletiva para que este conste da lista de referência. \\

\begin{tabular}{|l|c|} \hline
BRASIL. Ministério da Saúde. \textbf{Pesquisa nacional sobre saúde e nutri-} \\ \textbf{ção}: resultados preliminares e condições nutricionais da população brasileira: \\ adultos e idosos. Brasília, DF: IPEA, IBGE, INAN, 1990. 33 p.   \\\hline
\end{tabular}\\

\textbf{Campos em LATEX:}

\begin{verbatim}
@Book{brasil1990,
Title                    = {Pesquisa nacional sobre saúde e nutrição},
Address                  = {Brasília, DF},
Org-short                = {Brasil},
Organization             = {Brasil. {Ministério da Saúde}},
Pages                    = {33},
Publisher                = {IPEA, IBGE, INAN},
Subtitle                 = {resultados preliminares e condições 
nutricionais da população brasileira: adultos e idosos.},
Year                     = {1990},
Owner                    = {apcalabrez},
Timestamp                = {2015.09.22}
\end{verbatim}

Para este exemplo a citação será pela jurisdição geográfica \cite{brasil1990}. \\

\begin{tabular}{|l|c|} \hline
SÃO PAULO (Estado). Secretaria da Agricultura. \textbf{O café}: estatística de \\produção e commercio 1935-1936. São Paulo: Typ. Brasil de Rothschild, \\1937. 261 p.  \\\hline
\end{tabular}\\

\textbf{Campos em LATEX:}

\begin{verbatim}
@Book{saopaulo1937,
Title                    = {O café},
Address                  = {São Paulo},
Org-short                = {S\~ao Paulo},
Organization             = {S\~ao Paulo {(Estado). Secretaria da 
Agricultura}},
Pages                    = {261},
Publisher                = {Typ. Brasil de Rothschild},
Subtitle                 = {estatística de produção e commercio 1935-
1936.},
Year                     = {1937},
Owner                    = {apcalabrez},
Timestamp                = {2015.09.23}
}
\end{verbatim}

Para este exemplo a citação será pela jurisdição geográfica \cite{saopaulo1937}. \\

Em caso de duplicidade de nomes, deve-se acrescentar entre parêntese a unidade geográfica que identifica a jurisdição a que pertence. \\

\begin{tabular}{|l|c|} \hline
BIBLIOTECA NACIONAL (Brasil). \textbf{Movimento de vanguarda na Euro-} \\ \textbf{pa e modernismo brasileiro (1909-1924)}. Rio de Janeiro, 1976.	83 p.   \\\hline
\end{tabular}\\

\textbf{Campos em LATEX:}

\begin{verbatim}
@Book{bibliotecanacional1976,
Title                    = {Movimento de vangarda na Europa e modernismo
brasileiro (1909-1924)},
Address                  = {Rio de Janeiro},
Org-short                = {Biblioteca Nacional},
Organization             = {Biblioteca nacional {(Brasil)}},
Pages                    = {83},
Year                     = {1976},
Owner                    = {apcalabrez},
Timestamp                = {2015.09.23}
\end{verbatim}

\begin{tabular}{|l|c|} \hline
BIBLIOTECA NACIONAL (Portugal). \textbf{O 24 de Julho de 1833 e a} \\ \textbf{guerra civil de 1829-1834}. Lisboa, 1983. 95 p.   \\\hline
\end{tabular}\\

\textbf{Campos em LATEX:}

\begin{verbatim}
@Book{bibliotecanacional1983,
Title                    = {O 24 de Julho de 1833  e a guerra civil de 
1829-1834},
Address                  = {Lisboa},
Org-short                = {Biblioteca Nacional},
Organization             = {Biblioteca nacional {(Portugal)}},
Pages                    = {95},
Year                     = {1983},
Owner                    = {apcalabrez},
Timestamp                = {2015.09.23}
}
\end{verbatim}
\subsubsection{Autor(es) com mais de uma obra referenciada}

Quando se referenciam várias obras do mesmo autor, pode-se substituir
as seguintes por um traço sublinear (equivalente a seis espaços) e
ponto. 
No ~\autoref{quadro-opcoes-composicao-sublinear} estão descritos os comandos dos pacotes de alteração da composição dos estilos bibliográficos para alterar o estilo sublinear.

\begin{quadro}[H]
	\caption{\label{quadro-opcoes-composicao-sublinear}Opções de alteração da composição dos estilos bibliográficos para inserção de traço sublinear}
		\begin{tabular}{|p{4.0cm}|p{2.0cm}|p{8.5cm}|}
			\hline
			\textbf{Campo} & \textbf{Opções} & \textbf{Descrição} \\ 
			\hline
			\emph{abnt-repeated-author-omit} &   & Permite suprimir o autor que aparece
			repetidas vezes na sequência.\\
			\hline
			\emph{abnt-repeated-author-omit=no} & \emph{no} & Repete os autores. \\
			\hline
			\emph{abnt-repeated-author-omit=yes} & \emph{yes} & Substitui o autor repetido por \underline{\ \ \ \ \ \ \ \ }. \\
			\hline	
		\end{tabular}
	\begin{flushleft}
	Fonte: \citeonline{abnetxcite}
	\end{flushleft}	
\end{quadro}

Sendo assim, para que o traço sublinear conste da lista de referências deve-se acrescentar o pacote: 

\verb+\usepackage[alf,abnt-repeated-author-omit=yes]{abntex2cite}+

Para que ao criar as listas de referências as obras de mesmos autores sejam listadas conforme abaixo: \\

\begin{tabular}{|l|c|} \hline
	PICCINI, A. \textbf{Casa de Babylonia}: estudo da habitação rural no interior de \\São Paulo. São Paulo: Annablume, 1996. 165 p. \\
	
	\underline{\ \ \ \ \ \ \ \ }. \textbf{Cortiços na cidade}: conceito e preconceito na reestruturação do \\centro urbano de  São Paulo. São Paulo: Annablume, 1999. 166 p.   \\\hline
\end{tabular}\\

\textbf{Campos em LATEX:}

\begin{verbatim}
@Book{Piccini1996,
Title                    = {Casa de Babylonia},
Address                  = {São Paulo},
Author                   = {Piccini, A.},
Pages                    = {165},
Publisher                = {Annablume},
Subtitle                 = {estudo da habitação rural no interior de 
São Paulo},
Year                     = {1996},
Owner                    = {apcalabrez},
Timestamp                = {2015.09.23}
}
\end{verbatim}

\begin{verbatim}
Book{Piccini1999,
Title                    = {Cortiços na cidade},
Address                  = {São Paulo},
Author                   = {Piccini, A.},
Pages                    = {166},
Publisher                = {Annablume},
Subtitle                 = {conceito e preconceito na reestruturação do 
centro urbano de São Paulo},
Year                     = {1999},
Owner                    = {apcalabrez},
Timestamp                = {2015.09.21}
}

\end{verbatim}
\subsubsection{Mais de um volume}

\begin{tabular}{|l|c|} \hline
KUHN, H. A.; LASCH, H. G. \textbf{Avaliação clínica e funcional do doente}. \\São Paulo:  E.P.U., 1977. 4 v.    \\\hline
\end{tabular}\\

\textbf{Campos em LATEX:}

\begin{verbatim}
@Book{Kuhn1977,
Title                    = {Avaliação clínica e funcional do doente},
Address                  = {São Paulo},
Author                   = {Kuhn, H. A. and Lasch, H. G.},
Publisher                = {E. P. U.},
Year                     = {1977},
Volume                   = {4},
Owner                    = {apcalabrez},
Timestamp                = {2016.04.11}
}
\end{verbatim}
\subsubsection{Catálogo}

\begin{tabular}{|l|c|} \hline
BIBLIOTECA NACIONAL (Brasil). \textbf{500 anos de Brasil na Biblioteca }\\ \textbf{Nacional}: catálogo. Rio de Janeiro, 2000. 143 p. Catálogo da exposição em \\comemoração aos 500  anos do Brasil e aos 190 anos da Biblioteca Nacional, \\13 de dezembro de 2000 a 20 de abril de 2001.    \\\hline
\end{tabular}\\

\textbf{Campos em LATEX:}

\begin{verbatim}
@Book{bibliotecanacional2000,
Title                    = {500 anos de Brasil na Biblioteca Nacional},
Address                  = {Rio de Janeiro},
Note                     = {Catálogo da exposição em comemoração aos 500
anos do Brasil e aos 190 anos da Biblioteca Nacional, 13 de dezembro de
2000 a 20 de abril de 2001},
Org-short                = {Biblioteca Nacional},
Organization             = {Biblioteca Nacional {(Brasil)}},
Pages                    = {143},
Subtitle                 = {catálogo},
Year                     = {2000},
Owner                    = {apcalabrez},
Timestamp                = {2015.09.18}
}
\end{verbatim}

\begin{tabular}{|l|c|} \hline
DEMAKOPOULOU, K. et al. \textbf{Gods and heroes of the european}\\ \textbf{bronze age}. London:  Thames and Hudson, 2000. 303 p. Catalog.    \\\hline
\end{tabular}\\

\textbf{Campos em LATEX:}

\begin{verbatim}
@Book{Demakopoulou2000,
Title                    = {Gods and heroes of the european bronze age},
Address                  = {London},
Author                   = {Demakopoulou, K. and Arruda, M. L. and Souza,
L. S. and Saadi, S.},
Note                     = {Catalog},
Pages                    = {303},
Publisher                = {Thames and Hudson},
Year                     = {2000},
Owner                    = {apcalabrez},
Timestamp                = {2015.09.18}
}
\end{verbatim}
\subsubsection{Relatório e parecer técnico}

\begin{tabular}{|l|c|} \hline
	CASTRO, M. C. et al. \textbf{Cooperação técnica na implementação do
	Pro-}\\\textbf{grama Integrado de Desenvolvimento - Polonordeste}. Brasília:
	PNUD: \\FAO, 1990. 47 p. Relatório da Missão de Avaliação do Projeto
	BRA/87/037.     \\\hline
\end{tabular}\\

\textbf{Campos em LATEX:}

\begin{verbatim}
@Book{Castro,
Title                    = {Cooperação técnica na implementação do 
Programa Integrado 
de Desenvolvimento - Polonordeste},
Address                  = {Brasília},
Author                   = {Castro, M. C. and Souza, L. S. and Cardoso, 
R. F and Arruda, M. L.},
Note                     = {Relatório da Missão de Avaliação do 
Projeto BRA/87/037},
Pages                    = {47},
Publisher                = {PNUD: FAO},
Year                     = {1990},

Owner                    = {apcalabrez},
Timestamp                = {2015.09.17}
}
\end{verbatim}

\begin{tabular}{|l|c|} \hline
COMPANHIA ESTADUAL DE TECNOLOGIA DE SANEAMENTO AMBI-\\ENTAL. \textbf{Bacia hidrográfica do Ribeirão Pinheiros}: relatório técnico. São \\Paulo: CETESB, 1994. 39 p.   \\\hline
\end{tabular}\\

\textbf{Campos em LATEX:}

\begin{verbatim}
@Book{Castro,
@Book{CETESB1994,
Title                    = {Bacia hidrográfica do Ribeirão Pinheiros},
Address                  = {São Paulo},
Organization             = {Companhia Estadual de Tecnologia de 
Saneamento Ambiental},
Pages                    = {39},
Publisher                = {CETESB},
Subtitle                 = {relatório técnico},
Year                     = {1994},
Owner                    = {apcalabrez},
Timestamp                = {2015.09.17}
}
\end{verbatim}

\begin{tabular}{|l|c|} \hline
GUBITOSO, M. D. \textbf{Máquina worm}: simulador de máquinas paralelas. \\São Paulo: IME- USP, 1989. 29 p. Relatório técnico, Rt-Mac-8908.   \\\hline
\end{tabular}\\

\textbf{Campos em LATEX:}

\begin{verbatim}
@Book{Gubitoso1989,
Title                    = {Máquina worm},
Address                  = {São Paulo},
Author                   = {Gubitoso, M. D.},
Note                     = {Relatório técnico, Rt-Mac-8908},
Pages                    = {29},
Publisher                = {IME-USP},
Subtitle                 = {simulador de máquinas paralelas},
Year                     = {1989},
Owner                    = {apcalabrez},
Timestamp                = {2015.09.17}
\end{verbatim}

\subsubsection{Dicionário}

\begin{tabular}{|l|c|} \hline
DORLAND'S illustrated medical dictionary. 29th. ed. Philadelphia: W.\\B. Saunders, 2000.   \\\hline
\end{tabular}\\


\textbf{Campos em LATEX:}

\begin{verbatim}
@Book{Dorlands2000,
Title                    = {Dorland's illustrated medical dictionary},
Address                  = {Philadelphia},
Org-short                = {DORLAND'S},
Publisher                = {W.B. Saunders},
Year                     = {2000},
Edition                  = {29th.},
Owner                    = {apcalabrez},
Timestamp                = {2015.09.24}
               = {2015.09.17}
\end{verbatim}


\begin{tabular}{|l|c|} \hline
PACIORNICK, R. (Ed.). \textbf{Dicionário médico}. 3. ed. Rio de Janeiro: Gua-\\nabara Koogan,  1978.  \\\hline
\end{tabular}\\


\textbf{Campos em LATEX:}

\begin{verbatim}
@Book{Paciornick1978,
Title                    = {Dicionário médico},
Address                  = {Rio de Janeiro},
Editor                   = {Paciornick, R.},
Publisher                = {Guanabara Koogan},
Year                     = {1978},
Edition                  = {3.},
Owner                    = {apcalabrez},
Timestamp                = {2015.09.24}

\end{verbatim}
\subsubsection{Trabalhos acadêmicos}

Elementos essenciais: \\

\begin{tabular}{|l|c|} \hline
Autor, \textbf{título}, substítulo (se houver), data, número de folhas, grau, vincu-\\lação acadêmica, unidade de defesa, local, data de defesa e ano. \\\hline
\end{tabular}\\
\
 
Elementos complementares: Notas. \\

\begin{tabular}{|l|c|} \hline
SOBRENOME, Prenome do autor. \textbf{Título}: subtítulo. Data (ano de
depó-\\sito). Folhas. Grau de dissertação, tese, monografia ou
trabalho de conclu-\\são de curso - Unidade onde foi defendida,
Local, data (ano da defesa).\\\hline
\end{tabular}\\
\\

Exemplos \\

\begin{tabular}{|l|c|} \hline
ALMEIDA, G. A. \textbf{Resíduos de pesticida organoclorados no} \\ \textbf{complexo estuarino-lagunar Iguape-Cananéia e rio Ribeira}\\ \textbf{e Iguape}. 1995. 95 f.
Dissertação (Mestrado em Oceanografia Física)\\ - Instituto
Oceanográfico, Universidade de São Paulo, São Paulo, 1995.   \\\hline
\end{tabular} \\

\textbf{Campos em LATEX:}

\begin{verbatim}
@Mastersthesis{Almeida1995,
Title                    = {Resíduos de pesticida organoclorados 
no complexo estuarino-lagunar Iguape-Cananéia e rio Ribeira e Iguape},
Address                  = {São Paulo},
Author                   = {Almeida, G. A.},
Pagename                 = {f},
Pages                    = {95},
School                   = {Instituto Oceanográfico, Universidade de 
São Paulo},
Type                     = {Mestrado em Oceanografia Física},
Year                     = {1995},
Owner                    = {apcalabrez},
Timestamp                = {2015.09.23}
\end{verbatim}

\begin{tabular}{|l|c|} \hline
ALVES, J. M. \textbf{Competividade e tendência da produção de manga} \\ \textbf{para exportação do nordeste do Brasil}. 2002. 147 f. + 1 CD-ROM.\\ Tese (Doutorado em Economia Aplicada) - Escola Superior de Agricultura \\ "Luiz de Queiroz", Universidade de São Paulo, Piracicaba, 2002.    \\\hline
\end{tabular} \\

\textbf{Campos em LATEX:} 

\begin{verbatim}
@Phdthesis{Alves2002,
Title                    = {Competividade e tendência da produção de 
manga para exportação do nordeste do Brasil},
Address                  = {Piracicaba},
Author                   = {Alves, J. M.},
Pagename                 = {f. + 1 CD-ROM},
Pages                    = {147},
School                   = {Escola Superior de Agricultura "Luiz de 
Queiroz", Universidade de São Paulo},
Type                     = {Doutorado em Economia Aplicada},
Year                     = {2002},
Owner                    = {apcalabrez},
Timestamp                = {2015.09.23}
}
\end{verbatim} 

\begin{tabular}{|l|c|} \hline
DIAS, F. L. F. \textbf{Efeito da aplicação de calcário, lodo de esgoto e vinhaça} \\ \textbf{em solo cultivado em sorgo granífero (Sorghum bicolor L. Moench)}. \\1994. 74 f. Trabalho de Conclusão do Curso (Engenharia Agronômica) - Facul-\\dade de Ciências Agrárias e Veterinárias, Universidade Estadual Paulista \\"Júlio de Mesquita Filho", Jaboticabal, 1994.     \\\hline
\end{tabular} \\
	
	\textbf{Campos em LATEX:} 
	
	\begin{verbatim}
@Thesis{Dias1994,
Title                    = {Efeito da aplicação de calcário, lodo de esgoto 
e vinhaça 
em solo cultivado em sorgo granífero (Sorghum bicolor L. Moench)},
Address                  = {Jaboticabal},
Author                   = {Dias, F. L. F.},
Pagename                 = {f},
Pages                    = {74},
School                   = {Faculdade de Ciências Agrárias e Veterinárias, 
Universidade Estadual Paulista "Júlio de Mesquita Filho"},
Type                     = {Trabalho de Conclusão do Curso (Engenharia 
Agronômica)},
Year                     = {1994},
Owner                    = {apcalabrez},
Timestamp                = {2015.09.23}
}
	\end{verbatim}
\subsection{Parte de monografia}	

\begin{tabular}{|l|c|} \hline
SOBRENOME, Prenome(s) do(s) autor(es). Título do capítulo. In: SOBRENO-\\ME, Prenome(s) do(s) autor(es) da obra principal. \textbf{Título da obra}: subtítulo. \\Edição. Local: Editora, data de publicação.capítulo, p. inicial-final.     \\\hline
\end{tabular} \\

\subsubsection{Autor distinto da obra no todo} 

\begin{tabular}{|l|c|} \hline
CATANI, A. M. O que é capitalismo. In: SPINDEL, A. \textbf{Que é socialismo e o}\\ \textbf{que é comunismo}. São Paulo: Círculo do Livro, 1989. p. 7-87. (Primeiros \\passos, 1).    \\\hline
\end{tabular} \\

	\textbf{Campos em LATEX:} 
	
	\begin{verbatim}
@Incollection{Catani1989,
Title                    = {O que é capitalismo},
Author                   = {Catani, A. M.},
Booktitle                = {O que é socialismo e o que é comunismo},
Organization             = {Spindel, A.},
Publisher                = {Círculo do Livro},
Year                     = {1989},
Address                  = {São Paulo},
Note                     = {(Primeiros Passos, 1)},
Pages                    = {7-87},
Owner                    = {apcalabrez},
Timestamp                = {2015.09.25}
}
\end{verbatim}


\begin{tabular}{|l|c|} \hline
MOSS, D. W.; HENDERSON, A. R. Clinical enzymology. In: BURTIS, C. \\A.; ASHWOOD, E. R. (Ed.). \textbf{Tietz textbook of clinical chemistry}. 3rd\\ ed. Philadelphia: W. B. Saunders, 1999. cap. 22, p. 617-721.  \\\hline
\end{tabular} \\

\textbf{Campos em LATEX:} 

\begin{verbatim}
@Incollection{Moss1999,
Title                    = {Clinical enzymology},
Author                   = {Moss, D. W. and Henderson, A. R.},
Booktitle                = {Tietz textbook of clinical chemistry},
Publisher                = {W. B. Saunders},
Year                     = {1999},
Address                  = {Philadelphia},
Chapter                  = {22},
Edition                  = {3rd},
Editor                   = {Burtis, C. A. and Ashwood, E. R.},
Pages                    = {617-721},
Owner                    = {apcalabrez},
Timestamp                = {2015.09.25}
\end{verbatim}

\subsubsection{Mesmo autor da obra no todo}

Usam-se seis traços sublineares em substituição ao(s) nome(s) do(s) autor(es). \\
	 
\begin{tabular}{|l|c|} \hline
MONTGOMERY, R.; CONWAY, T. W.; SPECTOR, A. A. Estructuras de \\las proteínas.  In:\underline{\ \ \ \ \ \ \ \ }. \textbf{Bioquímica}: casos y texto. 5. ed. St. Louis:
	Mosby, \\1992. cap. 2, p. 41-90.  \\\hline
\end{tabular} \\ 

\textbf{Campos em LATEX:} 

\begin{verbatim}
@Inbook{Montgomery1992,
Title                    = {Estructuras de las proteínas},
Author                   = {Montgomery, R. and Conway, T. W. and 
Spector, 
A. A.},
Pages                    = {41-90},
Publisher                = {Mosby},
Year                     = {1992},
Address                  = {St. Louis},
Edition                  = {5},
Booksubtitle             = {casos y texto},
Booktitle                = {Bioquímica},
Chapter                  = {2},
Owner                    = {apcalabrez},
Timestamp                = {2015.09.25}
\end{verbatim}
	 
\begin{tabular}{|l|c|} \hline
RAMOS, M. E. M. Serviços administrativos na Bicen da UEPG. In:
\underline{\ \ \ \ \ \ \ \ }. \\ \textbf{Tecnologia e novas formas de gestão em bibliotecas
universitárias}. \\Ponta Grossa: UEPG, 1999. p. 157-182.   \\\hline
	 \end{tabular} \\ 
	 
	 \textbf{Campos em LATEX:} 
	 
\begin{verbatim}
@Inbook{Ramos1999,
Title                    = {Serviços administrativos na {Bicen da UEPG}},
Author                   = {Ramos, M. E. M.},
Pages                    = {157-182},
Publisher                = {UEPG},
Year                     = {1999},
Address                  = {Ponta Grossa},
Booktitle                = {Tecnologia e novas formas de gestão em 
bibliotecas universitárias},
Owner                    = {apcalabrez},
Timestamp                = {2015.09.25}
	 }
\end{verbatim}

\subsection{Monografia em suporte eletrônico}	 
	 
\begin{tabular}{|l|c|} \hline
SOBRENOME, Prenome(s) do(s) autor(es). \textbf{Título da obra}:
subtítulo.\\ Edição. Local: Editora, data de publicação. Disponível em: <endereço \\eletrônico>. Acesso em: dia mês abreviado ano.     \\\hline
\end{tabular} \\
	 
	Exemplos: \\ 
	 
\begin{tabular}{|l|c|} \hline
DUDEK, S. G. (Ed.). \textbf{Nutrition essentials for nursing practice}. \\5th ed. Philadelphia: Lippincott \& Williams  Wilkins, 2006. Disponível \\ em: <http://gateway.ut.ovid.com/gw1/ovidweb.cgi>. Acesso em: 24 out. \\2006.  \\\hline
\end{tabular} \\ 
	 
	 \textbf{Campos em LATEX:} 
	 
\begin{verbatim}
@Book{Dudek2006,
Title                    = {Nutrition essentials for nursing practice},
Address                  = {Philadelphia},
Editor                   = {Dudek, S. G.},
Publisher                = {Lippincott Williams \& Wilkins},
Year                     = {2006},
Edition                  = {5th},
Url                      = {http://gateway.ut.ovid.com/gw1/ovidweb.cgi},
Urlaccessdate            = {24 out. 2011},
Owner                    = {apcalabrez},
Timestamp                = {2015.09.28}

	 \end{verbatim}
	 
	 
	 \begin{tabular}{|l|c|} \hline
	 	NATIONAL RESEARCH COUNCIL. \textbf{Nutrient requirements of dairy }\\ \textbf{cattle}. 7th ed. Washington: National Academy of Sciences, 2001. 408 p.\\	Disponível em: <http:www.nap.edu/books/0309069971/html>. Acesso
	 	em: \\13 maio 2001.   \\\hline
	 \end{tabular} \\ 
	 
	 \textbf{Campos em LATEX:} 
	 
	 \begin{verbatim}
	@Book{council2001,
	Title                    = {Nutrient requirements of dairy cattle},
	Address                  = {Washington},
	Org-short                = {National Research Council},
	Organization             = {National Research Council},
	Pages                    = {408},
	Publisher                = {National Academy of Sciences},
	Year                     = {2001},
	Edition                  = {7th},
	Url                      = {http:www.nap.edu/books/0309069971/html},
	Urlaccessdate            = {13 maio 2001},
	Owner                    = {apcalabrez},
	Timestamp                = {2015.09.28}
	}
	 \end{verbatim}
	 
	 
	 	  \begin{tabular}{|l|c|} \hline
	 	 THOMÉ, V. M. R. et al. \textbf{Zoneamento agroecológico e socioeconômico do } \\ \textbf{Estado de Santa Catarina}:  versão preliminar. Florianópolis: EPAGRI, 1999. \\1 CD-ROM.  \\\hline
	 	 \end{tabular} \\ 
	 	 
	 	 \textbf{Campos em LATEX:} 
	 	 
	 	 \begin{verbatim}
	 	 @Book{Thome1999,
	 	 Title                    = {Zoneamento agroecológico e socioeconômico do 
	 	 Estado de Santa Catarina},
	 	 Address                  = {Florianópolis},
	 	 Author                   = {Thom\'e, V. M. R. and Souza, L. S. and 
	 	 Oliveira, A. P. and Silva, A. M.},
	 	 Note                     = {1 CD-ROM},
	 	 Publisher                = {EPAGRI},
	 	 Subtitle                 = {versão preliminar},
	 	 Year                     = {1999},
	 	 Owner                    = {apcalabrez},
	 	 Timestamp                = {2015.09.28}
	 	 }
	 	 \end{verbatim}
	 \subsubsection{Parte de monografia em suporte eletrônico}
	  	  
	   \begin{tabular}{|l|c|} \hline
	   	SOBRENOME, Prenome(s) do(s) autor(es). Título do capítulo. In:
	   	SOBRENO-\\ME, Prenome(s)  do(s) autor(es) da obra principal.  \textbf{Título da obra}: subtítulo. \\Edição. Local: Editora, data de publicação. capítulo, p. inicial-final. Disponível \\em: <endereço eletrônico>. Acesso em: dia mês abreviado ano.  \\\hline
	   \end{tabular} \\ 
	   
	   	Exemplos: \\ 
	   	
	   	\begin{tabular}{|l|c|} \hline
	   		SÃO PAULO (Estado). Secretaria do Meio Ambiente. Tratados e organizações\\ ambientais em matéria de meio ambiente. In: \underline{\ \ \ \ \ \ \ \ }. \textbf{Entendendo o meio} \\\textbf{ambiente}. São Paulo, 1999. v. 1. Disponível em: <http://www/bdf.org.br/\\sma/entendendo/atual.htm>. Acesso em: 9 mar.
	   		1999.  \\\hline
	   	\end{tabular} \\ 
	   	
	   	\textbf{Campos em LATEX:} 
	   	
	   	\begin{verbatim}
@Inbook{tratados1999,
Title                    = {Tratados e organizações ambientais em matéria 
de meio ambiente},
Org-short                = {S\~ao Paulo},
Organization             = {S\~ao Paulo {(Estado). Secretaria do Meio 
Ambiente}},
Url                      = {http://www/bdf.org.br/sma/entendendo/atual.
htm},
Urlaccessdate            = {9 mar. 1999},
Year                     = {1999},
Address                  = {São Paulo},
Volume                   = {1},
Booktitle                = {Entendendo o meio ambiente},
Or-short                 = {São Paulo},
Owner                    = {apcalabrez},
Timestamp                = {2015.09.28}
}
	   	\end{verbatim}
	   	
\begin{tabular}{|l|c|} \hline
ZELEN, M. Theory and practice of clinical trials. In: BAST Jr, R. C. \\et al. (Ed.). \textbf{Cancer medicine e.5.} Hamilton: BC Decker; New York: \\American Cancer Society, 2000. CD-ROM  \\\hline
\end{tabular} \\ 
	   		
\textbf{Campos em LATEX:} 
	   		
	   		
\begin{verbatim}
@Incollection{Zelen2000,
Title                    = {Theory and practice of clinical trials},
Author                   = {Zelen, M.},
Booktitle                = {Cancer medicine e.5},
Publisher                = {BC Decker},
Year                     = {2000},
Address                  = {Hamilton},
Editor                   = {Bast, J{r}, R. C. and Arruda, A. C. and 
Marques, A. P. and Oliveira, A. C.},
Note                     = {CD-ROM},
Owner                    = {apcalabrez},
Timestamp                = {2015.09.28}
}
\end{verbatim}
	   		
\subsection{Evento}
%\textbf{4.1.4 Evento} \\

Conjunto dos documentos reunidos em um produto final com denominação
de: atas, anais, proceedings, resumos entre outros. \\

\begin{tabular}{|l|c|} \hline
NOME DO EVENTO, numeração do evento em arábico (se
houver), ano, lo-\\cal de realização do evento. \textbf{Título do documento...} (Anais, Atas, \\Resumos etc.). Local: Editora, data de publicação. Páginas \\\hline
\end{tabular} \\ 

\subsubsection{Completo} 
	 
\begin{tabular}{|l|c|} \hline
ANNUAL MEETING OF THE AMERICAN SOCIETY OF INTERNATIO-\\NAL LAW, 65., 1967,  Washington. \textbf{Proceedings...} Washington: ASIL, 1967. \\227 p \\\hline
\end{tabular} \\

\textbf{Campos em LATEX:} 

\begin{verbatim}
@Proceedings{law1967,
Title                    = {Proceedings...},
Address                  = {Washington},
Conference-location      = {Washington},
Conference-number        = {65},
Conference-year          = {1997},
Organization             = {Annual Meeting of the American Society of 
International Law},
Pages                    = {227},
Publisher                = {ASIL},
Year                     = {1967},
Owner                    = {apcalabrez},
Timestamp                = {2015.09.28}
}
\end{verbatim}

\begin{tabular}{|l|c|} \hline
REUNIÃO ANUAL DA SOCIEDADE BRASILEIRA DE QUÍMICA, 20., \\1997, Poços de Caldas. \textbf{Química}: academia, indústria, sociedade: livro de \\resumos. São Paulo: Sociedade Brasileira de Química, 1997.  \\\hline
\end{tabular} \\

\textbf{Campos em LATEX:} 

\begin{verbatim}
@Proceedings{quimica1997,
Title                    = {Química},
Address                  = {São Paulo},
Conference-location      = {Poços de Caldas},
Conference-number        = {20},
Conference-year          = {1997},
Organization             = {Reuni\~ao Anual da Sociedade Brasileira de 
Qu{\'\í}mica},
Publisher                = {Sociedade Brasileira de Química},
Subtitle                 = {academia, indústria, sociedade: livro de 
resumos},
Year                     = {1997},
Owner                    = {apcalabrez},
Timestamp                = {2015.09.28}
}
\end{verbatim}
\subsubsection{Trabalho apresentado em evento}

\begin{tabular}{|l|c|} \hline
BRAYNER, A. R. A.; MEDEIROS, C. B. Incorporação do tempo em SGBD \\orientado a objetos. In: SIMPÓSIO BRASILEIRO DE BANCO DE DADOS, \\9., 1994, São Paulo. \textbf{Anais...} São Paulo: USP, 1994. p. 16-29.  \\\hline
\end{tabular} \\

\textbf{Campos em LATEX:} 

\begin{verbatim}
Title                    = {Incorporação do tempo em {SGBD} orientado a 
objetos},
Author                   = {Brayner, A. R. A. and Medeiros, C. B.},
Booktitle                = {Anais...},
Conference-location      = {São Paulo},
Conference-number        = {9},
Conference-year          = {1994},
Year                     = {1994},
Address                  = {São Paulo},
Organization             = {Simp\'osio Brasileiro de Banco de Dados},
Pages                    = {16-29},
Publisher                = {USP},
Owner                    = {Ana Paula},
Timestamp                = {2015.09.10}
}
\end{verbatim}

\subsubsection{Atas de conferências}

\begin{tabular}{|l|c|} \hline
KRONSTRAND, R. et al. Relationship between melanin and codeine
concen-\\trations in hair after oral administration. In: ANNUAL MEETINGS OF THE \\AMERICAN  ACADEMY OF FORENSIC SCIENCE, 1999, Orlando. \\\textbf{Proceedings…} Orlando:  Academic Press, 1999. p. 12.   \\\hline
\end{tabular} \\

\textbf{Campos em LATEX:} 

\begin{verbatim}
@Inproceedings{kronstrand1994,
Title                    = {Relationship between melanin and codeine
concentrations in hair after oral administration},
Author                   = {Kronstrand, R. and Arruda, M. L. and Kuhn, 
H. A. and Braams, J.},
Booktitle                = {Proceedings...},
Conference-location      = {Orlando},
Conference-year          = {1999},
Year                     = {1994},
Address                  = {Orlando},
Organization             = {Annual Meetings of the American Academy of 
Forensic Science},
Pages                    = {12},
Publisher                = {Academic Press},
Owner                    = {Ana Paula},
Timestamp                = {2015.09.10}
}
\end{verbatim}
\subsubsection{Trabalho de evento publicado em periódico} 

\begin{tabular}{|l|c|} \hline
	MINGRONI-NETTO, R. C. Origin of fmr-1 mutation: study of closely linked \\microsatellite loci in fragile x syndrome. \textbf{Brazilian Journal of Genetics}, \\Ribeirão Preto, v. 19, n.3, p. 144, 1996. Supplement. Program and abstract \\42nd. National Congress of Genetics, 1996. 
 \\\hline
\end{tabular} \\

\textbf{Campos em LATEX:} 

\begin{verbatim}
@Article{Mingroni-Netto1996,
Title                    = {Origin of fmr-1 mutation: study of closely 
linked microsatellite loci in fragile x syndrome},
Author                   = {Mingroni-Netto, R. C},
Journal                  = {Brazilian Journal of Genetics},
Year                     = {1996},
Address                  = {Ribeirão Preto},
Note                     = {Supplement. Program and abstract 42nd. 
National Congress of Genetics, 1996},
Number                   = {3},
Pages                    = {144},
Volume                   = {19},
Owner                    = {AnaPaula},
Timestamp                = {2015.10.02}
\end{verbatim} \\
\subsubsection{Evento em suporte eletrônico} 

\begin{tabular}{|l|c|} \hline
	NOME DO EVENTO, numeração do evento em arábico (se
	houver), ano, \\local de realização do evento. \textbf{Título do
	documento...} (Anais, Atas, \\Resumos etc.). Local: Editora, data de publicação. Disponível em: <endereço \\eletrônico>. Acesso em: dia mês abreviado. ano. 
	\\\hline
\end{tabular} \\

\textbf{Exemplo:} \\

\begin{tabular}{|l|c|} \hline
	SIMPÓSIO INTERNACIONAL DE INICIAÇÃO CIENTÍFICA DA
	UNIVER-\\SIDADE DE SÃO PAULO, 8., 2000, São Paulo. \textbf{Resumos...}
	São Paulo: USP, \\2000. 1 CD-ROM.  \\\hline
\end{tabular} \\

\textbf{Campos em LATEX:} 

\begin{verbatim}
@Proceedings{Simposio2000,
Title                    = {Resumos...},
Address                  = {São Paulo},
Conference-location      = {São Paulo},
Conference-number        = {8},
Conference-year          = {2000},
Organization             = {Simp\'osio Internacional de Iniciação 
Cient{\'\i}fica da Universidade de São Paulo},
Publisher                = {USP},
Year                     = {2000},
Note                     = {1 CD-ROM},
Owner                    = {apcalabrez},
Timestamp                = {2015.09.28}
\end{verbatim}

\subsubsection{Trabalho de evento em suporte eletrônico }

\begin{tabular}{|l|c|} \hline
SABROZA, P. C. Globalização e saúde: impacto nos perfis
epidemiológicos \\das populações. In: CONGRESSO BRASILEIRO DE
EPIDEMIOLOGIA, 4., \\1998, Rio de Janeiro. \textbf{Anais eletrônicos...} Rio de
Janeiro: ABRASCO, 1998. \\Mesa-redonda. Disponível em:
<http://www.abrasco.com.br/epino98/>. \\Acesso em: 17 jan. 1999.\\\hline 
\end{tabular} \\

\textbf{Campos em LATEX:} 

\begin{verbatim}
@Inproceedings{Sabroza1998,
Title                    = {Globalização e saúde},
Author                   = {Sabroza, P. C.},
Booktitle                = {Anais eletrônicos...},
Conference-location      = {Rio de Janeiro},
Conference-number        = {4},
Conference-year          = {1998},
Subtitle                 = {impacto nos perfis epidemiológicos das 
populações},
Year                     = {1998},
Address                  = {Rio Janeiro},
Note                     = {Mesa-redonda},
Organization             = {Congresso Brasileiro de Epidemiologia},
Publisher                = {ABRASCO},
Url                      = {http://www.abrasco.com.br/epino98/},
Urlaccessdate            = {17 jan. 1999},
Owner                    = {apcalabrez},
Timestamp                = {2015.10.01}
}
\end{verbatim}

\section{Publicações Periódicas}

Revistas, jornais, publicações anuais e séries monográficas, quando
tratadas como publicação periódica. \\

\subsection{Coleção como um todo}

\textbf{Exemplo:} \\

\begin{tabular}{|l|c|} \hline
NATURE. London, GB: Macmillan Magazines, 1869- . Semanal. ISSN
0028-\\0836.\\\hline
\end{tabular} \\

\textbf{Campos em LATEX:} 

\begin{verbatim}
@Journalpart{Nature1869,
Title                    = {Nature},
Address                  = {London, GB},
ISSN                     = {0028-0836},
Note                     = {Semanal},
Publisher                = {Macmillan Magazines},
Year                     = {1869-},
Owner                    = {apcalabrez},
Timestamp                = {2015.10.01}
}
\end{verbatim}

\subsection{Artigo de revista}


\begin{tabular}{|l|c|} \hline
BOYD, A. L.; SAMID, D. Molecular biology of transgenic animals. \textbf{Journal } \\ \textbf{of  Animal Science}, Albany, v. 71, n. 3, p. 1-9, 1993.
	\\\hline
\end{tabular} \\

\textbf{Campos em LATEX:} 

\begin{verbatim}
@Article{Boyd1993,
Title                    = {Molecular biology of transgenic animals},
Author                   = {Boyd, A. L and Samid, D.},
Journal                  = {Journal of Animal Science},
Year                     = {1993},
Address                  = {Albany},
Number                   = {3},
Pages                    = {1-9},
Volume                   = {71},
Owner                    = {apcalabrez},
Timestamp                = {2015.10.02}
}
\end{verbatim}

\begin{tabular}{|l|c|} \hline
KRAUSS, J. K. et al. Flow void of cerebrospinal fluid in idiopathic normal\\
pressure hydrocephalus of the elderly: can it predict outcome after
shunting? \\\textbf{Neurosurgery}, Baltimore, v. 40, n. 1, p. 67-73, 1997.
Discussion 73-74. 
	\\\hline
\end{tabular} \\

\textbf{Campos em LATEX:} 

\begin{verbatim}
@Article{Krauss1997,
Title                    = {Flow void of cerebrospinal fluid in idiopathic 
normal pressure hydrocephalus of the elderly:},
Author                   = {Krauss, J. K. and Souza, L. S. and Silva, A. M. 
and Arruda, M. L. and Mansilla, H. C. F.},
Journal                  = {Neurosurgery},
Subtitle                 = {can it predict outcome after shunting?},
Year                     = {1997},
Address                  = {Baltimore},
Note                     = {Discussion 73-74},
Number                   = {1},
Pages                    = {67-73},
Volume                   = {40},
Owner                    = {apcalabrez},
Timestamp                = {2015.10.02}
}
\end{verbatim}

\subsection{Editorial} 


\begin{tabular}{|l|c|} \hline
BRENNAN, R. J.; SONDORP, E. Humanitarian aid: some political realities. \\ \textbf{British Medical Journal}, London, v. 333, n. 7573, p. 817-818, out. 2006. \\Editorial. Disponível em: <http://bmj.bmjjournals.com/cgi/reprint/333/7573/\\817>. Acesso em: 24 out. 2006. \\\hline
\end{tabular} \\

\textbf{Campos em LATEX:} 

\begin{verbatim}
@Article{Brennan2006,
Title                    = {Humanitarian aid},
Author                   = {Brennan, R. J. and Sondorp, E.},
Journal                  = {British Medical Journal},
Subtitle                 = {some political realities},
Year                     = {2006},
Address                  = {London},
Month                    = {out.},
Note                     = {Editorial},
Number                   = {7573},
Pages                    = {817-818},
Url                      = {http://bmj.bmjjournals.com/cgi/reprint/333/
7573/817},
Urlaccessdate            = {24 out. 2006},
Volume                   = {333},
Owner                    = {apcalabrez},
Timestamp                = {2015.10.02}
}
\end{verbatim}

\begin{tabular}{|l|c|} \hline
COSTA, S. Os sertões: cem anos. \textbf{Revista USP}, São Paulo, v. 54, p. 5, jul./\\ago. 2002. Editorial.\\\hline
\end{tabular} \\

\textbf{Campos em LATEX:} 

\begin{verbatim}
@Article{Costa2002,
Title                    = {Os sertões},
Author                   = {Costa, S.},
Journal                  = {Revista USP},
Subtitle                 = {cem anos},
Year                     = {2002},
Address                  = {São Paulo},
Month                    = {jul./ago.},
Note                     = {Editorial},
Owner                    = {apcalabrez},
Timestamp                = {2015.10.02}
}
\end{verbatim}
\subsection{Entidade coletiva}

\begin{tabular}{|l|c|} \hline
COCHRANE INJURIES GROUP ALBUMIN REVIEWERS. Human \\albumin administration in critically ill patients: systematic review of \\randomized controlled trials. \textbf{British Medical} \textbf{Journal}, London, v. 317, \\n. 7153, p. 235-240, 1998. 
	\\\hline
\end{tabular} \\

\textbf{Campos em LATEX:} 

\begin{verbatim}
@Article{Cochrane1998,
Title                    = {Human albumin administration in critically 
ill patients:systematic review of randomized controlled trials.},
Journal                  = {British Medical Journal},
Org-short                = {Cochrane Injuries Group Albumin Reviewers},
Organization             = {Cochrane Injuries Group Albumin Reviewers},
Year                     = {1998},
Address                  = {London},
Number                   = {7153},
Pages                    = {235-240},
Volume                   = {317},
Owner                    = {apcalabrez},
Timestamp                = {2015.10.02}
}
\end{verbatim}

\subsection{Artigos em suplementos ou em números especiais}

\begin{tabular}{|l|c|} \hline
BOYD, A. L.; SAMID, D. Molecular biology of transgenic animals. \textbf{Journal } \\ \textbf{of Animal Science}, Albany, v. 71, p. 1-9, 1993. Supplement 3. 
	\\\hline
\end{tabular} \\

\textbf{Campos em LATEX:} 

\begin{verbatim}
@Article{Boyd1993,
Title                    = {Molecular biology of transgenic animals},
Author                   = {Boyd, A. L and Samid, D.},
Journal                  = {Journal of Animal Science},
Year                     = {1993},
Address                  = {Albany},
Note                     = {Supplement 3},
Pages                    = {1-9},
Volume                   = {71},
Owner                    = {apcalabrez},
Timestamp                = {2015.10.02}
}
\end{verbatim}

\begin{tabular}{|l|c|} \hline
HOOD, D. W. The utility of complete genome sequences in the study of \\pathogenic bacteria. \textbf{Parasitology}, Cambridge, v. 118, p. S3-S9, 1999. \\Supplement. \\\hline
\end{tabular} \\

\textbf{Campos em LATEX:} 

\begin{verbatim}
@Article{Hood1999,
Title                    = {The utility of complete genome sequences in 
the study of pathogenic bacteria},
Author                   = {Hood, D. W.},
Journal                  = {Parasitology},
Year                     = {1999},
Address                  = {Cambridge},
Note                     = {Supplement},
Pages                    = {S3-S9},
Volume                   = {118},
Owner                    = {apcalabrez},
Timestamp                = {2015.10.02}
}
\end{verbatim}

\begin{tabular}{|l|c|} \hline
TOLLIVET, M. Agricultura e meio ambiente: reflexões sociológicas. \\\textbf{Estudos Econômicos},  São Paulo, v. 24, p. 138-198, 1994. Número \\especial. 
	\\\hline
\end{tabular} \\

\textbf{Campos em LATEX:} 

\begin{verbatim}
@Article{Tollivet1994,
Title                    = {Agricultura e meio ambiente: reflexões 
sociológicas},
Author                   = {Tollivet, M},
Journal                  = {Estudos Econômicos},
Year                     = {1994},
Address                  = {São Paulo},
Note                     = {Número especial},
Pages                    = {138-198},
Volume                   = {24},
Owner                    = {apcalabrez},
Timestamp                = {2015.10.02}
}
\end{verbatim}
\subsection{Artigo publicado em partes}
%\textbf{4.2.6 Artigo publicado em partes} \\

\begin{tabular}{|l|c|} \hline
ABEND, S. M.; KULISH, N. The psychoanalytic method from an\\
epistemological viewpoint. \textbf{International Journal of Psycho-Analysis}, \\London, v. 83, pt. 2, p. 491-495, 2002. \\\hline
\end{tabular} \\

\textbf{Campos em LATEX:} 

\begin{verbatim}
@Article{Abend2002,
Title                    = {The psychoanalytic method from an 
epistemological viewpoint},
Author                   = {Abend, S. M. and Kulish},
Journal                  = {International Journal of Psycho-Analysis},
Year                     = {2002},
Address                  = {London},
Pages                    = {491-495},
Volume                   = {83, pt. 2},
Owner                    = {apcalabrez},
Timestamp                = {2015.10.02}
}
\end{verbatim}
\subsection{Artigo com errata publicada}

\begin{tabular}{|l|c|} \hline
MALINOWSKI, J. M.; BOLESTA, S. Rosiglitazone in the treatment of
type \\2 diabetes mellitus: a critical review. Clinical Therapetucis,
Princeton, v. 22, \\n. 10, p. 1151-1168, 2000. Errata em: \textbf{Clinical
Therapeutics}, Princeton, \\v. 23, n. 2, p. 309, 2001.
	\\\hline
\end{tabular} \\

\textbf{Campos em LATEX:} 

\begin{verbatim}
@Article{Malinowski2000,
Title                    = {Rosiglitazone in the treatment of type 
2 diabetes mellitus},
Author                   = {Malinowski, J. M and Bolesta, S.},
Journal                  = {Clinical Therapetucis},
Subtitle                 = {a critical review},
Year                     = {2000},
Address                  = {Princeton},
Note                     = {Errata em: \textbf{Clinical Therapeutics}, 
Princeton, v. 23, n. 2, p. 309, 2001},
Number                   = {10},
Pages                    = {1151-1168},
Volume                   = {22},
Owner                    = {apcalabrez},
Timestamp                = {2015.10.02}
}
\end{verbatim}
\subsection{Com indicação do mês}

\begin{tabular}{|l|c|} \hline
HARRISON, P. Update on pain management for advanced genitourinary	\\cancer. \textbf{Journal of Urology}, Baltimore, v. 165, n. 6, p. 1849-1858, June \\2001. 
	\\\hline
\end{tabular} \\

\textbf{Campos em LATEX:} 

\begin{verbatim}
@Article{Harrison2001,
Title                    = {Update on pain management for advanced 
genitourinary 
cancer},
Author                   = {Harrison, P.},
Journal                  = {Journal of Urology},
Year                     = {2001},
Address                  = {Baltimore},
Month                    = {June},
Number                   = {6},
Pages                    = {1849-1858},
Volume                   = {165},
Owner                    = {AnaPaula},
Timestamp                = {2015.10.02}
}
\end{verbatim}

\begin{tabular}{|l|c|} \hline
OLIVEIRA, R. et al. Preparações radiofarmacêuticas e suas aplicações.\\
\textbf{Revista Brasileira de Ciências Farmacêuticas}, São Paulo, v. 42, n. 2,\\
p. 151-165, abr./jun. 2006. \\\hline
\end{tabular} \\

\textbf{Campos em LATEX:} 

\begin{verbatim}
@Article{Oliveira2006,
Title                    = {Preparações radiofarmacêuticas e suas 
aplicações},
Author                   = {Oliveira, R. and Silva, A. M. and Arruda, 
M. L. and 
Malinowski, J. M},
Journal                  = {Revista Brasileira de Ciências Farmacêuticas},
Year                     = {2006},
Address                  = {São Paulo},
Month                    = {abr./jun.},
Number                   = {2},
Pages                    = {151-165},
Volume                   = {42},
Owner                    = {AnaPaula},
Timestamp                = {2015.10.02}
\end{verbatim}

\subsection{Artigo no prelo}

É considerado no prelo o artigo já aceito para publicação pelo Conselho
Editorial do periódico.

\textbf{Nota:} em português: No prelo, em inglês: In press, em alemão: In druck
e em francês: Sous press. 

\begin{tabular}{|l|c|} \hline
ELEWA, H. H. Water resources and geomorphological characteristics of
\\Tushka and west of Lake Nasser, Agypt. \textbf{Hydrogeology Journal}, Berlin,
\\v. 16, n. 1, 2006. In press. \\\hline
\end{tabular} \\

\textbf{Campos em LATEX:} 

\begin{verbatim}
@Article{Elewa2006,
Title                    = {Water resources and geomorphological 
characteristics of Tushka and west of Lake Nasser, Agypt},
Author                   = {Elewa, H. H.},
Journal                  = {Hydrogeology Journal},
Year                     = {2006},
Address                  = {Berlin},
Note                     = {In press},
Number                   = {1},
Volume                   = {16},
Owner                    = {AnaPaula},
Timestamp                = {2015.10.02}
}
\end{verbatim}

\begin{tabular}{|l|c|} \hline
PAULA, F. C. E. et al. Incinerador de resíduos líquidos e pastosos.
\textbf{Revista} \\ \textbf{ de Engenharia e Ciências Aplicadas}, São Paulo, v. 5, n. 2,
2001. No \\prelo. \\\hline
\end{tabular} \\

\textbf{Campos em LATEX:} 

\begin{verbatim}
@Article{Paula2001,
Title                    = {Incinerador de resíduos líquidos e pastosos},
Author                   = {Paula, F. C. E and Cardoso, R. F and Oliveira, 
A. P. and Silva, A. M. and Guimarães, P. C.},
Journal                  = {Revista de Engenharia e Ciências Aplicadas},
Year                     = {2001},
Address                  = {São Paulo},
Note                     = {No prelo},
Volume                   = {5},
Owner                    = {apcalabrez},
Timestamp                = {2015.09.16}
}
\end{verbatim}

\subsection{Publicações periódicas em suporte eletrônico}

\begin{tabular}{|l|c|} \hline
SOBRENOME, Prenome(s) do(s) autor(es). Título do artigo: subtítulo. \ \\\textbf{Título}  \textbf{da publicação}, Local de publicação (cidade), volume, fascículo, \\paginação inicial e final do artigo e mês abreviado de publicação. Dispo-\\nível em: <endereço eletrônico>. Acesso em: dia mês abreviado ano. \\\hline
\end{tabular} \\

\textbf{Exemplos:} \\

\begin{tabular}{|l|c|} \hline
PALAGACHEV, D. K.; RECKE, L.; SOFTOVA, L. G. Applications of\\ the
differential calculus to nonlinear elliptic operators with discontinuous\\
coefficients.  \textbf{Mathematische Annalen}, Berlin, v. 336, n. 3, p. 617-637,
\\Nov. 2006. Disponível em:
<http://www.springerlink.com.w10077.dotlib.\\com.br/content/y767134777
841722/fulltext.pdf>. Acesso em: 17 nov. \\2006. 
	\\\hline
\end{tabular} \\

\textbf{Campos em LATEX:} 

\begin{verbatim}
Title                    = {Applications of the differential calculus 
to nonlinear
elliptic operators with discontinuous coefficients.},
Author                   = {Palagachev, D. K. and Recke, L and 
Softova, 
L. G.},
Journal                  = {Mathematische Annalen},
Year                     = {2006},
Address                  = {Berlin},
Month                    = {nov.},
Number                   = {3},
Pages                    = {617-637},
Url                      = {http://www.springerlink.com.w10077.dotlib.
com.br/content/y767134777841722/fulltext.pdf},
Urlaccessdate            = {17 nov. 2006},
Volume                   = {336},
Owner                    = {AnaPaula},
Timestamp                = {2015.10.02}
}
\end{verbatim}

\begin{tabular}{|l|c|} \hline
WU, H. et al. Parametric sensitivity in fixed-bed catalytic reactors with \\
reverse flow operation. \textbf{Chemical Engineering Science}, London, v. 54,\\
n. 20, 1999. Disponível em: <http://www.probe.br/sciencedirect.html>. \\Acesso em: 8 nov. 1999. \\\hline
\end{tabular} \\

\textbf{Campos em LATEX:} 

\begin{verbatim}
@Article{Wu1999,
Title                    = {Parametric sensitivity in fixed-bed 
catalytic reactors with reverse flow operation},
Author                   = {Wu, H. and Silva, A. M. and Montgomery, 
R. and Arruda, M. L.},
Journal                  = {Chemical Engineering Science},
Year                     = {1999},
Address                  = {London},
Number                   = {20},
Url                      = {http://www.probe.br/sciencedirect.html},
Urlaccessdate            = {8 nov. 1999},
Volume                   = {54},
Owner                    = {AnaPaula},
Timestamp                = {2015.10.02}
}
\end{verbatim}
\subsection{Artigo e/ou matéria de jornal}

\begin{tabular}{|l|c|} \hline
HOFLING, E. Livro descreve os 134 tipos de aves do campus da USP. \textbf{O} \\ \textbf{Estado de S. Paulo}, São Paulo, 15 out. 1993. Cidades, Caderno 7, p. 15. \\Depoimento a Luiz Roberto de Souza Queiroz.	\\\hline
\end{tabular} \\

\begin{verbatim}
@Article{Hofling1993,
Title                    = {Livro descreve os 134 tipos de aves do campus
da USP},
Author                   = {Hofling, E.},
Journal                  = {O Estado de S. Paulo},
Year                     = {1993},
Address                  = {São Paulo},
Month                    = {15 out.},
Note                     = {Cidades, Caderno 7, p. 15. Depoimento a Luiz 
Roberto de Souza Queiroz},
Owner                    = {AnaPaula},
Timestamp                = {2015.10.02}
}
\end{verbatim}

\textbf{-- Em suporte eletrônico} \\

\begin{tabular}{|l|c|} \hline
PORTER, E. This time, it's not the economy. \textbf{The New York Times}, \\New 
York, 24 Oct. 2006. Disponível em: <http://www.nytimes.com/2006\\/10/24/
business/usinessoref=slogin>. Acesso em: 24 out. 2006. \\\hline
\end{tabular} \\

\textbf{Campos em LATEX:} 

\begin{verbatim}
@Article{Porter2006,
Title                    = {This time, it's not the economy},
Author                   = {Porter, E.},
Journal                  = {The New York Times},
Year                     = {2006},
Address                  = {New York},
Month                    = {24 Oct.},
Url                      = {http://www.nytimes.com/2006/10/24/
business/usinessoref=slogin},
Urlaccessdate            = {24 out. 2006},
Owner                    = {AnaPaula},
Timestamp                = {2015.10.02}
}
\end{verbatim}

\subsection{Artigo publicado com correção}

\textbf{-- correção de} \\

\begin{tabular}{|l|c|} \hline
MEYAARD, L. et al. The epithelial celular adhesion molecule (Ep-CAM)\\
is a ligand for the leukocyte-associated immunoglobulin-like receptor
\\(LAIR). \textbf{Journal of Experimental Medicine}, New York, v. 198, n. 7,\\ 
p.	1129, Oct. 2003. Correção de: MEYAARD, L. et al. Journal of Experi-\\mental Medicine, New York, v. 194, n. 1, p. 107-112, July 2001.\\\hline
\end{tabular} \\

\textbf{Campos em LATEX:} 

\begin{verbatim}
@Article{Meyaard2003,
Title                    = {The epithelial celular adhesion molecule 
(Ep-CAM) is a ligand for the leukocyte-associated immunoglobulin-like 
receptor (LAIR).}, 
Author                   = {Meyaard, L and Arruda, M. L. and Silva, 
A. M. and Montgomery, R. and Malinowski, J. M},
Journal                  = {Journal of Experimental Medicine},
Year                     = {2003},
Address                  = {New York},
Month                    = {Oct.},
Note                     = {Correção de: MEYAARD, L. et al. Journal of 
Experimental Medicine, New York, v. 194, n. 1, p. 107-112, July 2001},
Number                   = {7},
Pages                    = {1129},
Volume                   = {198},
Owner                    = {AnaPaula},
Timestamp                = {2015.10.02}
}
\end{verbatim}

\textbf{-- correção em} \\

\begin{tabular}{|l|c|} \hline
MEYAARD, L. et al. The epithelial celular adhesion molecule (Ep-CAM)
\\is a ligand for the leukocyte-associated immunoglobulin-like receptor
(LAIR). \\Journal of Experimental Medicine, New York, v. 194, n. 1, p. 107-112, July \\2001. Correção em: MEYAARD, L. et al. \textbf{Journal of Experimental}\\ \textbf{Medicine}, New York, v. 198, n. 7, p. 1129, Oct. 2003. 
	\\\hline
\end{tabular} \\

\textbf{Campos em LATEX:} 

\begin{verbatim}
@Article{Meyaard2003,
Title                    = {The epithelial celular adhesion molecule 
(Ep-CAM) is a ligand for the leukocyte-associated immunoglobulin-like 
receptor (LAIR).},
Author                   = {Meyaard, L and Arruda, M. L. and Silva, 
A. M. and 
Montgomery, R. and Malinowski, J. M},
Journal                  = {Journal of Experimental Medicine},
Year                     = {2001},
Address                  = {New York},
Month                    = {July},
Note                     = {Correção em: MEYAARD, L. et al. 
\textbf{Journal of Experimental Medicine}, New York, v. 198, n. 7, 
p. 1129, Oct. 2003.},
Number                   = {1},
Pages                    = {107-112},
Volume                   = {194},
Owner                    = {AnaPaula},
Timestamp                = {2015.10.02}
}
\end{verbatim}

\section{Patentes}

\begin{tabular}{|l|c|} \hline
ENTIDADE RESPONSÁVEL. Nome do Autor/inventor na ordem \\direta. \textbf{Título}. Número da patente, datas (período de registro).
	\\\hline
\end{tabular} \\

\textbf{Exemplos:} \\

\begin{tabular}{|l|c|} \hline
EMBRAPA. Unidade de Apoio, Pesquisa e Desenvolvimento de
\\Instrumentação Agropecuária (São Carlos, SP). Paulo Estevão \\Cruvinel. \textbf{Medidor digital de temperatura para solos}. \\BR n. PI 8903105-9, 26 jun. 1989, 30 maio 1995. 
	\\\hline
\end{tabular} \\

\textbf{Campos em LATEX:} 

\begin{verbatim}
@Patent{Cruviel2014,
Title                    = {Medidor digital multisensorial de 
temperatura para solos},
Author                   = {Paulo Estev\~ao Cruvinel},
HowPublished             = {26 jul. 1989, 30 maio 1995},
Number                   = {BR1O 2014 0890310-5A2},
Organization             = {Embrapa. {Unidade de Apoio a 
Pesquisa e desenvolvimento de Instrumentação Agropecuária 
(São Carlos)}},
Owner                    = {Ana Paula},
Timestamp                = {2015.08.31}
}
\end{verbatim}

\begin{tabular}{|l|c|} \hline
MINOLTA COMPANY (Japan). Tomoko Miyaura. \textbf{Method for}\\ \textbf{manufacturing optical lens} \textbf{elements}. US 5720791A, 7 Mar. \\1995, 24
Feb. 1998. 
	\\\hline
\end{tabular} \\

\textbf{Campos em LATEX:} 

\begin{verbatim}
@Patent{Miyaura,
Title                    = {Method for manufacturing optical 
lens elements},
Author                   = {Tomoko Miyaura},
HowPublished             = {7 mar. 1995, 24 fev. 1998},
Number                   = {US 5720791A},
Organization             = {Minolta Company {(Japan)}},
Owner                    = {apcalabrez},
Timestamp                = {2015.09.15}
}
\end{verbatim}

\begin{tabular}{|l|c|} \hline
UNIVERSIDADE DE SÃO PAULO. Escola Politécnica. Waldir Pó.\\
\textbf{Conversor eletrônico de lâmpadas}. BR n. PI 6500856, 19 maio \\1985. 
	\\\hline
\end{tabular} \\

\textbf{Campos em LATEX:} 

\begin{verbatim}
@Patent{po1995,
Title                    = {Conversor eletrônico de lâmpadas},
Author                   = {Waldir P\'o},
HowPublished             = {19 maio 1985},
Number                   = {BR n. PI 6500856},
Organization             = {UNIVERSIDADE DE 
SÃO PAULO. {Escola Politécnica}},
Owner                    = {apcalabrez},
Timestamp                = {2015.09.15}
}
\end{verbatim}

\textbf{-- Em suporte eletrônico } \\

\begin{tabular}{|l|c|} \hline
ENTIDADE RESPONSÁVEL. Nome do Autor/inventor na ordem direta. \\\textbf{Título}. Número da patente, datas (período de registro). Disponível em: \\<endereço eletrônico>. Acesso em: dia mês abreviado. Ano. 
	\\\hline
\end{tabular} \\

\textbf{Exemplos:} \\

\begin{tabular}{|l|c|} \hline
IMPERIAL CHEMICAL INDUSTRIES PLC (London). David Ronald \\Hodgson; Francis Rourke. \textbf{Cathode for use in electrolyte cell}. US \\6017430, 6 Aug. 1997, 25 Jan. 2000. Disponível em:
<http://164.195.\\100.11/netacgi/nphParser?Sect1=PTO2Sect2=HITTOFFp1u=/neta\\html/srchnum.htmr=1f=Gl=5 Os1=6017430x>. Acesso em: 4 dez. \\2001. 
	\\\hline
\end{tabular} \\

\textbf{Campos em LATEX:} 

\begin{verbatim}
@Patent{imperial2000,
Title                    = {Cathode for use in electrolyte cell},
Author                   = {David Ronald Hodgson and Francis Rourke.},
HowPublished             = {25 Jan. 2000},
Number                   = {US 6017430},
Organization             = {Imperial Chemical Industries Plc (London).},
Url                      = {<http://164.195.100.11/netacgi/nphParser?Sect1
=PTO2Sect2=HITTOFFp=1u=/netahtml/srchnum.htmr=1f=Gl=5 O s1=6017430x>},
Urlaccessdate            = {4 dez. 2001},
Owner                    = {apcalabrez},
Timestamp                = {2015.09.15}
}

\end{verbatim}


\begin{tabular}{|l|c|} \hline
UNILEVER N. V. Elza Maria Possinhas Pimentel. \textbf{Dove}. BR n. PI 06520430,\\ 10 mar. 1977, 19 ago. 1997. Disponível em: <http://www.inpi.gov.br/pesqmar\\cas/ marcas.htm>. Acesso em: 30 abr.
2002. 
	\\\hline
\end{tabular} \\

\textbf{Campos em LATEX:} 

\begin{verbatim}
@Patent{unilever1997,
Title                    = {Dove},
Author                   = {Elza Maria Possinhas Pimentel},
HowPublished             = {10 mar. 1977, 19 ago. 1997},
Number                   = {BR n. PI 006520430},
Organization             = {Unilever N. V.},
Url                      = {<http://www.inpi.gov.br/pesq_marcas/
marcas.htm>},
Urlaccessdate            = {30 abr. 2002},
Owner                    = {apcalabrez},
Timestamp                = {2015.09.15}
}
\end{verbatim}

\section{Normas}

Norma é o documento estabelecido por consenso e aprovado por um organismo reconhecido, que fornece regras, diretrizes ou características mínimas para atividades ou para seus resultados, visando à obtenção de um grau ótimo de ordenação em um dado contexto.

\textbf{Exemplos:} \\

\begin{tabular}{|l|c|} \hline
ASSOCIAÇÃO BRASILEIRA DE NORMAS TÉCNICAS. \textbf{NBR 10520}: \\informação e documentação: citações em documentos: apresentação. Rio \\de Janeiro, 2002a. 7 p. 
	\\\hline
\end{tabular} \\

\textbf{Campos em LATEX:} 

\begin{verbatim}
@Book{nbr10520,
Title                    = {NBR 10520},
Address                  = {Rio de Janeiro},
Org-short                = {Associa{\c c}\~ao Brasileira de Normas 
T\'ecnicas},
Organization             = {Associa{\c c}\~ao Brasileira de Normas 
T\'ecnicas},
Pages                    = {7},
Subtitle                 = {informação e documentação: citações 
em documentos: 
apresentação},
Year                     = {2002a},
Owner                    = {apcalabrez},
Timestamp                = {2015.10.16}
}
\end{verbatim}

\begin{tabular}{|l|c|} \hline
INSTITUTO BRASILEIRO DE GEOGRAFIA E ESTATÍSTICA. \textbf{Normas} \\ \textbf{de apresentação tabular}. 3. ed. Rio de Janeiro, 1993. 
	\\\hline
\end{tabular} \\

\textbf{Campos em LATEX:} 

\begin{verbatim}
@Book{ibge1993,
Title                    = {Normas de apresentação tabular},
Address                  = {Rio de Janeiro},
Organization             = {Instituto Brasileiro de Geografia e 
Estat{\'\i}stica},
Publisher                = {IBGE},
Year                     = {1993},
Edition                  = {3},
Owner                    = {Ana Paula},
Timestamp                = {2015.09.10}
}

\end{verbatim}
\section{Documentos Jurídicos}

Documentos referentes à legislação, jurisprudência (decisões judiciais) e
doutrina (interpretação dos textos legais).
Elementos essenciais: jurisdição (ou cabeçalho da entidade, no caso de
se tratar de normas), título, numeração, data e dados da publicação. No
caso de constituições e suas emendas, entre com o nome da jurisdição, o
título e acrescente a palavra “Constituição”, seguida do ano de
promulgação, entre parênteses.
Elementos complementares: Notas explicativas. \\

\subsection{Legislação}

Compreende a Constituição, as emendas constitucionais e os textos
legais intraconstitucionais (lei complementar e ordinária, medida
provisória, decreto em todas as suas formas, resolução do Senado
Federal) e normas emanadas de entidades públicas e privadas (ato
normativo, portaria, resolução, ordem de serviço, instrução normativa,
comunicado, aviso, circular, decisão administrativa, entre outros). 

\textbf{Exemplos:} \\

\begin{tabular}{|l|c|} \hline
BRASIL. \textbf{Código civil}. Organização dos textos, notas remissivas e índices por \\Juarez de Oliveira. 46. ed. São Paulo: Saraiva, 1985. 
	\\\hline
\end{tabular} \\

\textbf{Campos em LATEX:} 
\begin{verbatim}
@Book{codigo1995,
Title                    = {Código civil},
Address                  = {São Paulo},
Furtherresp              = {Organização dos textos, notas remissivas e 
índices por 
Juarez de Oliveira},
Org-short                = {Brasil},
Organization             = {Brasil},
Publisher                = {Saraiva},
Year                     = {1985},
Edition                  = {46},
Owner                    = {AnaPaula},
Timestamp                = {2015.10.02}
}

\end{verbatim}

\begin{tabular}{|l|c|} \hline
BRASIL. \textbf{Constituição (1988)}. Constituição da República Federativa do \\Brasil. Brasília, DF: Senado, 1988. 
	\\\hline
\end{tabular} \\

\textbf{Campos em LATEX:} 

\begin{verbatim}
@Book{constituicao1988,
Title                    = {Constituição (1988)},
Address                  = {Brasília, DF},
Furtherresp              = {Constituição da República Federativa 
do Brasil.},
Org-short                = {Brasil},
Organization             = {Brasil},
Publisher                = {Senado},
Year                     = {1988},
Owner                    = {AnaPaula},
Timestamp                = {2015.10.02}
}
\end{verbatim}

\begin{tabular}{|l|c|} \hline
BRASIL. Constituição (1988). Emenda Constitucional 
nº 9, de 9 de \\novembro de 1995. Dá nova redação ao art. 177 da Constituição
Federal,\\ alterando e inserindo parágrafos. \textbf{Lex}, São Paulo, v. 59, p. 1966, \\out./dez. 1995.  
	\\\hline
\end{tabular} \\

\textbf{Campos em LATEX:} 

\begin{verbatim}
@Article{brasil1995,
Title                    = {Constituição (1988). Emenda constitucional 
nº 9, de 9 de novembro de 1995. Dá nova redação ao art. 177 da 
Constituição Federal, alterando e inserindo paragráfos},
Journal                  = {Lex},
Organization             = {Brasil},
Year                     = {1995},
Address                  = {São Paulo},
Volume                   = {59},
Owner                    = {Ana Paula},
Timestamp                = {2015.09.10}
}
\end{verbatim}

\begin{tabular}{|l|c|} \hline
BRASIL. Medida provisória nº 1.569-9, de 11 de dezembro de 1997.\\
Estabelece multa em operações de importação, e dá outras
providências.\\ \textbf{Diário Oficial [da] República Federativa do Brasil},
Poder Executivo, \\Brasília, DF, 14 dez. 1997. Seção 1, p. 29514.  
	\\\hline
\end{tabular} \\

\textbf{Campos em LATEX:} 

\begin{verbatim}
@Article{brasil1997,
Title                    = {Medida provisória nº 1.569-9, de 11 de 
dezembro de 1997. Estabelece multa em operações de importação, e 
dá outras providências},
Journal                  = {Diário Oficial da República Federativa 
do Brasil},
Organization             = {Brasil},
Year                     = {1997},
Address                  = {Brasília, DF},
Month                    = {14 dez.},
Note                     = {Seção 1, p.29514},
Owner                    = {Ana Paula},
Publisher                = {Poder Executivo},
Timestamp                = {2015.09.10}
}

\end{verbatim}
\subsection{Jurisprudência}

Súmulas, enunciados, acórdãos, sentenças e demais decisões judiciais. 

\textbf{Exemplos:} \\

\begin{tabular}{|l|c|} \hline
BRASIL. Tribunal Regional Federal. (5. Região). Administrativo. Escola\\
Técnica Federal. Pagamento de diferenças referente a enquadramento de \\
servidor decorrente da implantação de Plano Único de Classificação e \\
Distribuição de Cargos e Empregos, instituído pela Lei nº 8.270/91. \\
Predominância da lei sobre a portaria. Apelação cível nº 42.441-PE \\
(94.05.01629-6). Apelante: Edilemos Mamede dos Santos e outros. Ape-\\
lada: Escola Técnica Federal de Pernambuco. Relator: Juiz Nereu San-\\
tos. Recife, 4 de março de 1997. \textbf{Lex}: jurisprudência do STJ e Tribu-\\
nais Regionais Federais, São Paulo. v. 10, n.103, p. 558-562, mar. 1998. \\\hline
\end{tabular} \\

\textbf{Campos em LATEX:} 

\begin{verbatim}
@Article{brasillex1998,
Title                    = {Tribunal Regional Federal. Regi\~ao, 5. 
Administrativo. Escola T\’ecnica Federal. Pagamento de diferen{\c c}as 
referente a enquadramento de servidor decorrente de implanta{\c c}\~ao 
de Plano {{\’U}}nico de Classifica{\cc}\~ao e Distribui{\c c}\~ao de 
Cargos e Empregos, institu{\’\i}do pela Lei n{$^o$}~8.270/91. 
Predomin\^ancia da lei sobre a portaria. Apela{\cc}\~ao c{\’\i}vel
n{$^o$}~42.441-{PE} (94.05.01629-6). Apelante: Edilemos Mamede dos Santos
e outros. Apelada: Escola T\’ecnica Federal de Pernambuco. Relator: Juiz
Nereu Santos. Recife, 4 de mar{\c c}o de 1997},
Journal                  = {Lex},
Organization             = {Brasil},
Year                     = {1998},
Address                  = {S\~ao Paulo},
Month                    = {mar.},
Number                   = {103},
Pages                    = {558-562},
Volume                   = {10},
Section                  = {Jurisprud\^encia do STJ e Tribunais Regionais 
Federais}
Owner                    = {Ana Paula},
Timestamp                = {2015.09.10}
}
\end{verbatim}
\subsection{Doutrina}

Qualquer discussão técnica sobre questões legais (monografias, artigos
de periódicos, papers etc.), referenciada conforme o tipo de publicação. 

\textbf{Exemplos:} \\

\begin{tabular}{|l|c|} \hline
BARROS, Raimundo Gomes de. Ministério Público: sua legitimação	frente ao\\
Código do Consumidor. \textbf{Revista Trimestral de Jurisprudência dos}\\
\textbf{Estados}, São Paulo, v. 19, n. 139, p. 53-72, ago. 1995. \\\hline
\end{tabular} \\

\textbf{Campos em LATEX:} 

\begin{verbatim}
@Article{barros1995,
Title                    = {Ministério Público},
Author                   = {Barros, Raimundo Gomes de},
Journal                  = {Revista Trimestral de Jurisprudência dos 
Estados},
Subtitle                 = {sua legitimação
frente ao Código do Consumidor},
Year                     = {1995},
Address                  = {São Paulo,},
Month                    = {ago},
Number                   = {139},
Pages                    = {53-72},
Volume                   = {19},
Owner                    = {apcalabrez},
Timestamp                = {2016.04.26}
}
\end{verbatim}
\subsection{Em suporte eletrônico}
%\textbf{4.5.4 Em suporte eletrônico} \\

\begin{tabular}{|l|c|} \hline
BRASIL. Lei nº 9.887, de 7 de dezembro de 1999. Altera a legislação tributária\\
federal. \textbf{Diário Oficial [da] República Federativa do Brasil}, Brasília,\\
DF, 8 dez. 1999. Disponível em: <http://www.in.gov.br/mpleis/leistexto.asp?\\
ld=LEI209887>. Acesso em: 22 dez. 1999. 
\\\hline
\end{tabular} \\

\textbf{Campos em LATEX:} 

\begin{verbatim}
@Article{1999,
Title                    = {Lei nº 9.887, de 7 de dezembro de 1999. Altera 
a legislação tributária federal},
Journal                  = {Diário Oficial da República Federativa do 
Brasil},
Organization             = {Brasil},
Year                     = {1999},
Address                  = {Brasília, DF},
Month                    = {8 dez.},
Url                      = {http://www.in.gov.br/mp_leis/leis_texto.aps?
Id=Lei209887},
Urlaccessdate            = {22 dez. 1999},
Owner                    = {Ana Paula},
Timestamp                = {2015.09.10}
}
\end{verbatim}

\section{Materiais especiais}

Filmes cinematográficos ou científicos, gravações de vídeo e som,
esculturas, maquetes, objetos de museu, animais empalhados, jogos,
modelos, protótipos etc. \\

\begin{tabular}{|l|c|} \hline
	TÍTULO. Diretor, produtor. Local: Produtora, data. Especificação do	suporte\\
	em unidades físicas. Notas complementares. \\
	
	ou\\	
	
	SOBRENOME, Prenome(s) do(s) autor(es). \textbf{Título} (quando não 	existir,\\
	deve-se atribuir uma denominação ou a indicação sem 	título, entre col-\\
	chetes). Ano. Especificação do objeto. 
	\\\hline
\end{tabular} \\

\textbf{Exemplos:} \\

\begin{tabular}{|l|c|} \hline
BULE de porcelana: família Rosa, decorado com buquês e guirlandas de flores\\ 
sobre fundo branco, pegador de tampa em formato de fruto. [China: Compa-\\
nhia das Índias, 18--]. 1 bule.  
	\\\hline
\end{tabular} \\

\textbf{Campos em LATEX:} 

\begin{verbatim}
@Book{bule18,
Title                    = {Bule de porcela},
Note                     = {[China: Companhia das Índias, 18--]. 1 
bule.}, 
Org-short                = {Bule, 18--},
Subtitle                 = {família Rosa, decorado com buquês e 
guirlandas de flores sobre fundo branco, pegador de tampa em formato de 
fruto},
Owner                    = {apcalabrez},
Timestamp                = {2015.10.08}
}
\end{verbatim}

\begin{tabular}{|l|c|} \hline
CENTRAL do Brasil. Direção: Walter Salles Júnior. Produção: Martire de\\
Clermont-Tonnerre e Arthur Cohn. Intérpretes: Fernanda Montenegro; Ma-\\
rília Pera; Vinicius de Oliveira; Sônia Lira; Othon Bastos; Matheus\\ 
Nachtergaele e outros. Roteiro: Marcos Bernstein, João Emanuel Carnei-\\
ro e Walter Salles Júnior. [S.l.]: Le Studio Canal; Riofilme; MACT \\
Productions, 1998. 1 bobina cinematográfica (106 min), son., color., 
\\35 mm. 
	\\\hline
\end{tabular} \\

\textbf{Campos em LATEX:} 

\begin{verbatim}
@Book{central1998,
Title                    = {Central do Brasil},
Address                  = {[S.l.]},
Furtherresp              = {Direção: Walter Salles Júnior. Produção: 
Martire de Clermont-Tonnerre e Arthur Cohn. Intérpretes: Fernanda 
Montenegro; Marília Pera; Vinicius de Oliveira; Sônia Lira; Othon 
Bastos; Matheus Nachtergaele e outros. Roteiro: Marcos Bernstein, 
João Emanuel Carneiro e Walter Salles Júnior},
Note                     = {1 bobina cinematográfica (106 min), 
son., color., 35 mm},
Org-short                = {Central},
Publisher                = {Le Studio Canal; Riofilme; MACT 
Productions},
Year                     = {1998},
Owner                    = {apcalabrez},
Timestamp                = {2015.10.08}
}
\end{verbatim}


\begin{tabular}{|l|c|} \hline
KOBAYASHI, K. \textbf{Doença dos xavantes}. 1980. 1 fotografia, color., 16 cm x \\
56 cm. 
	\\\hline
\end{tabular} \\

\textbf{Campos em LATEX:} 

\begin{verbatim}
@Book{Kobayashi1980,
Title                    = {Doença dos xavantes},
Author                   = {Kobayashi, K.},
Note                     = {1 fotografia, color., 16 cm x 56 cm},
Year                     = {1980},
Owner                    = {apcalabrez},
Timestamp                = {2015.10.08}
}

Ou

@Misc{KOBAYASHI1980,
Title                    = {Doenças dos xavantes},
Author                   = {Kobayashi, K.},
Note                     = {1 fot., color. 16 cm X 56 cm.},
Year                     = {1980},
Owner                    = {Ana Paula},
Timestamp                = {2015.09.10}
}
\end{verbatim}
\subsection{Documentos Cartográficos}

Mapa, atlas, globo, fotografia aérea, imagem de satélite etc. 
\subsubsection{No todo}

\begin{tabular}{|l|c|} \hline
SOBRENOME, Prenome(s) do(s) autor(es). \textbf{Título}: subtítulo. Local: \\
Editora, ano, designação específica e escala
	\\\hline
\end{tabular} \\

\textbf{Exemplos:} \\

\begin{tabular}{|l|c|} \hline
ATLAS Mirador Internacional. Rio de Janeiro: Enciclopédia Britânica do\\
Brasil, 1981. 1 atlas. Escalas variam. 
	\\\hline
\end{tabular} \\

\textbf{Campos em LATEX:} 

\begin{verbatim}
@Book{atlas1981,
Title                    = {Atlas Mirador Internacional},
Address                  = {Rio de Janeiro},
Note                     = {1 atlas. Escalas variam},
Org-short                = {Atlas},
Publisher                = {Enciclopédia Britânica do Brasil},
Year                     = {1981},
Owner                    = {apcalabrez},
Timestamp                = {2015.10.08}
}
\end{verbatim}

\begin{tabular}{|l|c|} \hline
BRASIL e parte da América do Sul: mapa político, escolar, rodoviário,
turís-\\ 
tico e regional. São Paulo: Michalany, 1981. 1 mapa, color., 79 cm x\\
95 cm. Escala 1:600. 
	\\\hline
\end{tabular} \\

\textbf{Campos em LATEX:} 

\begin{verbatim}
@Book{brasil1981,
Title                    = {Brasil e parte da América do Sul},
Address                  = {São Paulo},
Note                     = {1 mapa, color., 79 cm x 95 cm. Escala 1:600},
Org-short                = {Brasil},
Publisher                = {Michalany},
Subtitle                 = {mapa político, escolar, rodoviário, turístico 
e regional},
Year                     = {1981},
Owner                    = {apcalabrez},
Timestamp                = {2015.10.08}
}
\end{verbatim}

\subsubsection{Em suporte eletrônico}

\begin{tabular}{|l|c|} \hline
SOBRENOME, Prenome(s) do(s) autor(es). \textbf{Título}: subtítulo. Local: Editora,\\
ano, designação específica e escala. Disponível em: <endereço eletrônico>. \\
Acesso em: dia mês abreviado. Ano. 
	\\\hline
\end{tabular} \\

\textbf{Exemplos:} \\

\begin{tabular}{|l|c|} \hline
ATLAS ambiental da Bacia do Rio Corumbataí. Rio Claro: CEAPLA, IGCE,\\
UNESP, 2001. Disponível em: <http://www.rc.unesp.br/igce/ceapla/atlas>.\\
Acesso em: 8 abr. 2002. 
	\\\hline
\end{tabular} \\

\textbf{Campos em LATEX:} 

\begin{verbatim}
@Book{atlas2001,
Title                    = {Atlas ambiental da Bacia do Rio Corumbataí},
Address                  = {Rio Claro},
Org-short                = {Atlas},
Publisher                = {CEAPLA, IGCE, UNESP},
Year                     = {2001},
Url                      = {http://www.rc.unesp.br/igce/ceapla/atlas},
Urlaccessdate            = {8 abr. 2002},
Owner                    = {apcalabrez},
Timestamp                = {2015.10.08}
}
\end{verbatim}

\subsection{Documentos sonoros}

Discos, CD, fita cassete, fita magnética etc. \\
\subsubsection{No todo}

\begin{tabular}{|l|c|} \hline
COMPOSITOR(ES) OU INTÉRPRETE(S). \textbf{Título}. Local: Gravadora, ano. \\
Especificação do suporte. 
	\\\hline
\end{tabular} \\

\textbf{Exemplos:} \\

\begin{tabular}{|l|c|} \hline
FAGNER, R. \textbf{Revelação}. Rio de Janeiro: CBS, 1988. 1 cassete sonoro (60 \\
min), 3 3/4 pps, estéreo.  
	\\\hline
\end{tabular} \\

\textbf{Campos em LATEX:} 

\begin{verbatim}
@Book{Fagner1988,
Title                    = {Revelação},
Address                  = {Rio de Janeiro},
Author                   = {Fagner, R.},
Note                     = {1 cassete sonoro (60 min), 3 3/4 pps, 
estéreo},
Publisher                = {CBS},
Year                     = {1988},
Owner                    = {AnaPaula},
Timestamp                = {2015.10.08}
}
\end{verbatim}

\begin{tabular}{|l|c|} \hline
DENVER, John. \textbf{Poems, prayers \& promises}. São Paulo: RCA Records, 1974.\\
1 disco (38 min): 33 1/3 rpm, microssulco, estéreo. 104.4049.   
	\\\hline
\end{tabular} \\

\textbf{Campos em LATEX:} 

\begin{verbatim}
@Book{Denver1974,
Title                    = {Poems, prayers \& promises},
Address                  = {São Paulo},
Author                   = {Denver, John},
Note                     = {1 disco (38 min): 33 1/3 rpm, microssulco, 
estéreo. 104.4049},
Publisher                = {RCA records},
Year                     = {1974},
Owner                    = {apcalabrez},
Timestamp                = {2015.10.08}
}
\end{verbatim}
\subsubsection{Em parte}
%\textbf{4.6.2.2 Em parte} \\

\begin{tabular}{|l|c|} \hline
COSTA. S.; SILVA, A. Jura secreta. Intérprete: Simone. In: SIMONE. \textbf{Face}\\ \textbf{a face}. [S.l.]: Emi-Odeon Brasil, p1977. 1 CD. Faixa 7. 
	\\\hline
\end{tabular} \\

\textbf{Campos em LATEX:} 

\begin{verbatim}
@Incollection{simone1977,
Title                    = {Jura secreta. Intérprete: Simone},
Author                   = {Costa, S and Silva, A.},
Booktitle                = {Face a face},
Org-short                = {Simone},
Organization             = {Simone},
Publisher                = {Emi-Odeon Brasil},
Year                     = {1977},
Address                  = {[S.l.]},
Note                     = {1 CD. Faixa 7},
Owner                    = {apcalabrez},
Timestamp                = {2015.10.08}
}
\end{verbatim}

\subsection{Partituras}

\subsubsection{Impressa}


\begin{tabular}{|l|c|} \hline
SOBRENOME, Prenome do autor. \textbf{Título}: subtítulo. Local: Editora, ano.\\
Designação do material (unidades físicas: número de partituras ou de partes,\\
páginas e/ou folhas). Instrumento a que se destina. 
	\\\hline
\end{tabular} \\

\textbf{Exemplos:} \\

\begin{tabular}{|l|c|} \hline
VILLA-LOBOS, H. \textbf{Coleções de quartetos modernos}: cordas. Rio de \\Janeiro: [s.n.], 1916. 1 partitura [23 p.]. Violoncelo. 
	\\\hline
\end{tabular} \\

\textbf{Campos em LATEX:} 

\begin{verbatim}
@Book{Villa-Lobos1916,
Title                    = {Coleções de quartetos modernos:},
Address                  = {Rio de Janeiro},
Author                   = {Villa-Lobos, H.},
Note                     = {1 partitura [23 p.]. Violoncelo},
Publisher                = {[s.n.]},
Subtitle                 = {cordas},
Year                     = {1916},
Owner                    = {apcalabrez},
Timestamp                = {2015.10.08}
}
\end{verbatim}

\subsubsection{Em suporte eletrônico}

\begin{tabular}{|l|c|} \hline
	SOBRENOME, Prenome do autor. \textbf{Título}: subtítulo. Local: Editora,
	ano. \\Designação do material (unidades físicas: número de
	partituras ou de partes).\\Instrumento a que se destina. Disponível
	em: <endereço eletrônico>. Acesso \\em: dia mês abreviado. Ano. 
	\\\hline
\end{tabular} \\

\textbf{Exemplos:} \\

\begin{tabular}{|l|c|} \hline
OLIVA, Marcos; MOCOTÓ, Tiago. \textbf{Fervilhar}: frevo. [19--?]. 1 partitura.
\\Piano. Disponível em: <http://openlink.inter.net/picolino/partitur.htm>.
\\Acesso: 5 jan. 2002. 
	\\\hline
\end{tabular} \\

\textbf{Campos em LATEX:} 

\begin{verbatim}
@Book{Oliva1900,
Title                    = {Fervilhar},
Author                   = {Oliva, M. and Mocot\'o, T.},
Note                     = {1 partitura. Piano},
Subtitle                 = {frevo},
Year                     = {[1900},
Url                      = {http://openlink.inter.net/picolino/partitur.
htm},
Urlaccessdate            = {5 jan. 2002},
Owner                    = {apcalabrez},
Timestamp                = {2015.10.08}
}
\end{verbatim}

\subsection{Bula de medicamento}

\begin{tabular}{|l|c|} \hline
TÍTULO da medicação. Responsável técnico (se houver). Local: Laboratório, \\ano de fabricação. Bula de remédio. 
	\\\hline
\end{tabular} \\

\textbf{Exemplos:} \\

\begin{tabular}{|l|c|} \hline
RESPRIN: comprimidos. Responsável técnico Delosmar R. Bastos. São José \\dos Campos: Johnson \& Johnson, 1997. Bula de remédio. 
	\\\hline
\end{tabular} \\

\textbf{Campos em LATEX:} 

\begin{verbatim}
@Book{resprin1997,
Title                    = {Resprin},
Address                  = {São José dos Campos},
Furtherresp              = {Responsável técnico Delosmar R. Bastos},
Note                     = {Bula de remédio},
Publisher                = {Johnson \& Johnson},
Subtitle                 = {comprimidos},
Year                     = {1997},
Owner                    = {apcalabrez},
Timestamp                = {2015.09.14}
}
\end{verbatim}

\textbf{-- Em suporte eletrônico} \\

\begin{tabular}{|l|c|} \hline
BUSCOPAN: composto. Responsável Técnico Dímitra Apostolopoulou.\\ Itacerica da Serra: Boehringer Ingelheim Brasil, 2013. Bula de remédio. \\Disponível em:<http://www.buscopan.com.br/content/dam/internet/\\chc/buscopan/pt-BR/documents/bula-buscopan-composto-comprimidos-\\revestidos-paciente.pdf>. Acesso em: 14 set. 2015.
	\\\hline
\end{tabular} \\

\textbf{Campos em LATEX:} 

\begin{verbatim}
@Book{buscopan2013,
Title                    = {Buscopan},
Address                  = {Itacerica da Serra},
Furtherresp              = {Responsável Técnico Dímitra 
Apostolopoulou},
Note                     = {Bula de remédio},
Publisher                = {Boehringer Ingelheim Brasil},
Subtitle                 = {composto},
Year                     = {2013},
Url                      = {http://www.buscopan.com.br/
content/dam/internet/chc/buscopan/pt_BR/documents/bula_
buscopan_composto_comprimidos_revestidos_paciente.pdf},
Urlaccessdate            = {14 set. 2015},
Owner                    = {apcalabrez},
Timestamp                = {2015.09.14}
}

\end{verbatim}

\section{Documentos disponíveis somente em suporte eletrônico}

Documento codificado para manipulação (edição, leitura) por computador, com acessos:\\
\textbf{• direto}: leitura efetuada por equipamentos periféricos ligados ao
computador (disquete, arquivos em disco rígido, CD-ROM, DVD);\\
\textbf{• remoto}: redes locais ou externas (banco e bases de dados,
catálogos ou livro, websites, serviços on-line, tais como: listas de
discussão, mensagens eletrônicas, arquivos etc.) \cite{Weitzc2016} \\

\begin{tabular}{|l|c|} \hline
SOBRENOME, Prenome(s). \textbf{Título} e versão (se houver) e descrição física \\do meio eletrônico. Quando se tratar de obras consultadas on-line, incluir o \\ endereço eletrônico. Disponível em: <endereço eletrônico>. Acesso em: dia \\mês abreviado. Ano. 
	\\\hline
\end{tabular} \\

\subsection{Acesso a banco, base de dados e lista de discussão}

\textbf{Exemplos:} \\

\begin{tabular}{|l|c|} \hline
ÁCAROS no Estado de São Paulo (Enseius concordis): banco de dados\\ preparado por Carlos H.W. Flechtmann. In: FUNDAÇÃO TROPICAL \\DE PESQUISAS E TECNOLOGIA "ANDRÉ TOSELLO". \textbf{Base de Dados} \\\textbf{Tropical}: no ar desde 1985. Disponível em: <http://www.bdt.org/bdt/aca\\rosp>. Acesso em: 28 nov. 1998. 
	\\\hline
\end{tabular} \\

\textbf{Campos em LATEX:} 

\begin{verbatim}
@Incollection{acaros1985,
Title                    = {Ácaros no Estado de São Paulo (Enseius 
concordis)},
Booksubtitle             = {no ar desde 1985},
Booktitle                = {Base de Dados Tropical},
Org-short                = {Ácaros},
Organization             = {Funda{\c c}\~ao Tropical De Pesquisas e 
Tecnologia "Andr\'e Tosello},
Subtitle                 = {banco de dados preparado por Carlos 
H.W. Flechtmann.},
Url                      = {http://www.bdt.org/bdt/acarosp.},
Urlaccessdate            = {28 nov. 1998},
Owner                    = {apcalabrez},
Timestamp                = {2015.10.08}
}
\end{verbatim}

\begin{tabular}{|l|c|} \hline
BIONLINE Discussion List. List maintained by the Bases de Dados Tropical, \\BDT in Brasil. Disponível em: <lisserv@bdt.org.br>. Acesso em: 25 nov.\\1998. 
	\\\hline
\end{tabular} \\

\textbf{Campos em LATEX:} 

\begin{verbatim}
@Book{bionline,
Title                    = {Bionline Discussion List.  
List maintained by the Bases de Dados Tropical, BDT in Brasil.},
Org-short                = {Bionline},
Url                      = {<lisserv@bdt.org.br},
Urlaccessdate            = {25 nov. 1998},
Owner                    = {apcalabrez},
Timestamp                = {2015.10.08}
}

\end{verbatim}

\begin{tabular}{|l|c|} \hline
UNIVERSIDADE DE SÃO PAULO. Sistema Integrado de Bibliotecas. \\\textbf{DEDALUS}: banco de dados bibliográficos da USP. São Paulo, 2006. \\Disponível em: <http://www.usp.br/sibi>. Acesso em: 16 out. 2006. 
	\\\hline
\end{tabular} \\

\textbf{Campos em LATEX:} 

\begin{verbatim}
@Book{usp2006,
Title                    = {Dedalus},
Address                  = {São Paulo},
Org-short                = {Universidade de S\~ao Paulo},
Organization             = {Universidade de S\~ao Paulo. {Sistema 
Integrado de 
Bibliotecas}},
Subtitle                 = {banco de dados bibliográficos da USP},
Year                     = {2006},
Url                      = {http://www.usp.br/sibi},
Urlaccessdate            = {16 out. 2006},
Owner                    = {apcalabrez},
Timestamp                = {2015.10.08}
}
\end{verbatim}

\subsection{Website}

\textbf{Exemplos:} \\

\begin{tabular}{|l|c|} \hline
GALERIA virtual de arte do Vale do Paraíba. São José dos Campos: \\Fundação Cultural Cassiano Ricardo, 1998. Apresenta reproduções \\virtuais de obras de artistas plásticos do Vale do Paraíba. Disponível \\em: <http://www.virtualvale.com.br/galeria>. Acesso em: 27 nov. \\2001. 
	\\\hline
\end{tabular} \\

\textbf{Campos em LATEX:} 

\begin{verbatim}
@Book{galeria1998,
Title                    = {Galeria virtual de arte do Vale da 
Paraíba},
Address                  = {São José dos Campos},
Note                     = {Apresenta reproduções virtuais de obras
de artistas 
plásticos do Vale do Paraíba},
Org-short                = {Galeria},
Publisher                = {Fundação Cultural Cassiano Ricardo},
Year                     = {1998},
Url                      = {http://www.virtualvale.com.br/galeria},
Urlaccessdate            = {27 nov. 2001},
Owner                    = {apcalabrez},
Timestamp                = {2015.10.08}
}

\end{verbatim}
\subsection{Artigo ahead of print}

Artigo aceito para publicação e disponível on-line, antes da impressão,
sem ter um número de fascículo associado. \\

\textbf{Exemplos:} \\

\begin{tabular}{|l|c|} \hline
SIGH-MANOUX, A.; RICHARDS, M.; MARMOT, M. Socieconomic
\\position acroos the lifecourse: how does is relate to cognitive function \\in
mid-life? \textbf{Annals of Epidemiology}, New York, 2005. In press. \\Disponível
em: <http://www.science.direct.com/science?-ob=Article\\URL>. Acesso em: 13 jan. 2005. 
	\\\hline
\end{tabular} \\

\textbf{Campos em LATEX:} 

\begin{verbatim}
@Article{Sigh-Manoux2005,
Title                    = {Socieconomic position acroos the 
lifecourse},
Author                   = {Sigh-Manoux, A. and Richrads, M. 
and Marmot, M.},
Journal                  = {Annals of Epidemiology},
Subtitle                 = {how does is relate to cognitive function 
in mid-life?},
Year                     = {2005},

Address                  = {New York},
Note                     = {In press},
Url                      = {<http://www.sciencedirect.com/science?
-ob=ArticleURL>},
Urlaccessdate            = {13 jan. 2005},
Owner                    = {apcalabrez},
Timestamp                = {2016.04.26}
}
\end{verbatim}

\begin{tabular}{|l|c|} \hline
TEIXEIRA JÚNIOR, A. L.; CARAMELLI, P. Apatia na doença de \\Alzheimer. \textbf{Revista Brasileira de Psiquiatria}, São Paulo, 2006. No\\ prelo. Disponível em:
<http://www.scielo.br/pdf/rbp/nahead/ahead1b.\\pdf>. Acesso em: 8 ago.
2006. 
	\\\hline
\end{tabular} \\

\textbf{Campos em LATEX:} 

\begin{verbatim}
@Article{Teixeira2006,
Title                    = {Apatia na doença de Alzheimer},
Author                   = {Teixeira, Junior, A. L. and Caramelli, 
P.},
Journal                  = {Revista Brasileira de Psiquiatria},
Year                     = {2006},
Address                  = {São Paulo},
Note                     = {No prelo},
Url                      = {<http://www.scielo.br/pdf/rbp/nahead
/ahead1b.pdf>},
Urlaccessdate            = {8 ago. 2006},
Owner                    = {apcalabrez},
Timestamp                = {2016.04.26}
}
\end{verbatim}

\subsection{Open access}

\textbf{Exemplo} \\

\begin{tabular}{|l|c|} \hline
LACASSE, J. R.; LEO, J. Serotonin and depression: a disconnect between \\ the advertisements and the scientific literature. \textbf{Plos Medicine}, San \\Francisco, v. 2, n. 12, p. e392, Dec. 2005. \emph{Open access}. Disponível em: \\<http://www.plosmedicine.org>. Acesso em: 15 mar. 2006. 
	\\\hline
\end{tabular} \\

\textbf{Campos em LATEX:} 

\begin{verbatim}
@Article{Lacasse2005,
Title                    = {Serotonin and depression: a disconnect 
between the 
advertisements and the scientific literature},
Author                   = {Lacasse, J. R. and Leo, J.},
Journal                  = {Plos Medicine},
Year                     = {2005},
Address                  = {San Francisco},
Month                    = {Dec.},
Note                     = {\emph{Open access}},
Number                   = {12},
Pages                    = {e392},
Url                      = {http://www.plosmedicine.org},
Urlaccessdate            = {15 mar. 2006},
Volume                   = {2},
Owner                    = {apcalabrez},
Timestamp                = {2015.10.08}
}
\end{verbatim}

\subsection{Digital Object Identifier (DOI)}

Representa um sistema de identificação numérico para localizar e
acessar materiais na web (publicações em periódicos, livros etc.), muitas
das quais localizadas em bibliotecas virtuais. Foi desenvolvido pela
Associação de Publicadores Americanos (AAP) com a finalidade de autenticar a base administrativa de conteúdo digital. Este número de
identificação da obra é composto por duas sequências: um prefixo (ou
raiz) que identifica o publicador do documento e um sufixo determinado
pelo responsável pela publicação do documento. \cite{Doic2016}.

Por exemplo: 34.7111.9 / ISBN (ou ISSN).

O prefixo DOI é nomeado pela International DOI Foundation (IDF),
garantindo identidade única a cada documento. \\


\begin{tabular}{|l|c|} \hline
SUKIKARA, M. H. et al. Opiate regulation of behavioral selection during \\lactation. \textbf{Pharmacology, Biochemistry and Behavior}, Phoenix, v. 87,\\ p. 315-320, 2007. doi:10.1016/j.pbb.2007.05.005. 
	\\\hline
\end{tabular} \\

\textbf{Campos em LATEX:} 

\begin{verbatim}
@Article{Sukikara2007,
Title                    = {Opiate regulation of behavioral selection 
during lactation},
Author                   = {Sukikara, M. H. and Arruda, M. L. and 
Softova,
L. G. and Malinowski, J. M},
Journal                  = {Pharmacology, Biochemistry and Behavior},
Year                     = {2007},
Address                  = {Phoenix},
Note                     = {doi:10.1016/j.pbb.2007.05.005},
Pages                    = {315-320},
Volume                   = {87},
Owner                    = {apcalabrez},
Timestamp                = {2015.10.08}
}

\end{verbatim}

\subsection{CD-ROM e disquete}

\textbf{Exemplo} \\

\begin{tabular}{|l|c|} \hline
MICROSOFT Project for Windows 95: project planning software. Version \\4.1. [S.l]: Microsoft Corporation, 1995. 1 CD-ROM. 
	\\\hline
\end{tabular} \\

\textbf{Campos em LATEX:} 

\begin{verbatim}
@Book{microsoft1995,
Title                    = {Microsoft Project for Windows 95},
Note                     = {1 CD-ROM},
Org-short                = {Microsoft},
Publisher                = {Microsoft Corporation},
Subtitle                 = {project planning software. Version 4.1.},
Year                     = {1995},
Owner                    = {apcalabrez},
Timestamp                = {2015.10.08}
}
\end{verbatim}

\subsection{Mensagens eletrônicas}

\textbf{Exemplo} \\

\begin{tabular}{|l|c|} \hline
SCIENCEDIRECT MESSAGE CENTER. \textbf{ScienceDirect Search Alert}: \\34 New articles Available on ScienceDirect [mensagem pessoal]. Mensagem \\recebida por <mjkarval@usp.br> em 17 nov. 2006. 
	\\\hline
\end{tabular} \\

\textbf{Campos em LATEX:} 

\begin{verbatim}
@Book{science2006,
Title                    = {ScienceDirect Search Alert},
Note                     = {Mensagem recebida por <mjkarval@usp.br> 
em 17 nov. 2006},
Org-short                = {Sciencedirect Message Center},
Organization             = {Sciencedirect Message Center},
Subtitle                 = {34 New articles Available on 
ScienceDirect [mensagem pessoal]},
Owner                    = {apcalabrez},
Timestamp                = {2015.10.08}
}
\end{verbatim}

As referências das citações presentes no capítulo "Referências"  também servem de exemplos para elaboração de bibliografia em BibTeX e constam do arquivo.bib. 

