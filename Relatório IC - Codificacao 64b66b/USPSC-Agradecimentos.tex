%% Agradecimentos.tex
% ---
% Agradecimentos
% ---=====
\begin{agradecimentos}
	A motivação para o desenvolvimento da classe USPSC e dos modelos de trabalhos acadêmicos foi decorrente de solicitações de usuários das Bibliotecas do Campus USP de São Carlos. A versão 2.0 do Pacote USPSC é composto da \textbf{Classe USPSC}, do \textbf{Modelo para TCC em \LaTeX\ utilizando a classe USPSC} e do \textbf{Modelo para teses e dissertações em \LaTeX\ utilizando a classe USPSC}.
	
	O Modelo para TCC está disponível inicialmente apenas para EESC e será estendido às demais Unidades de Ensino do Campus USP de São Carlos a medida que as mesmas definirem seus padrões.
	
	O Grupo desenvolvedor do Pacote USPSC agradece especialmente ao Luis Olmes, doutorando do Instituto de Ciências Matemáticas e de Computação (ICMC) da Universidade de São Paulo (USP), pelas primeiras orientações sobre o \LaTeX\ . 
	
	Agradecemos ao Lauro César Araujo pelo desenvolvimento da classe  \abnTeX, modelos canônicos e tantas outras contribuições que nos permitiu o desenvolvimento da classe USPSC e seus modelos.
	
	Os nossos agradecimentos aos integrantes do primeiro
	projeto abn\TeX\, Gerald Weber, Miguel Frasson, Leslie H. Watter, Bruno Parente Lima, Flávio de Vasconcellos Corrêa, Otavio Real
	Salvador, Renato Machnievscz, e a todos que contribuíram para que a produção de trabalhos acadêmicos em conformidade com
	as normas ABNT com \LaTeX\ fosse possível.
	
	Agradecemos ao grupo de usuários
	\emph{latex-br}{\url{http://groups.google.com/group/latex-br}}, aos integrantes do grupo
	\emph{\abnTeX}{\url{http://groups.google.com/group/abntex2}  e \url{http://www.abntex.net.br/}}~que contribuem para a evolução do \abnTeX.
\end{agradecimentos}
% ---